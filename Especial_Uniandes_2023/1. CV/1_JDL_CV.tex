%!TEX TS-program = xelatex
%!TEX encoding = UTF-8 Unicode
% Awesome CV LaTeX Template for CV/Resume
%
% This template has been downloaded from:
% https://github.com/posquit0/Awesome-CV
%
% Author:
% Claud D. Park <posquit0.bj@gmail.com>
% http://www.posquit0.com
%
%
% Adapted to be an Rmarkdown template by Mitchell O'Hara-Wild
% 23 November 2018
%
% Template license:
% CC BY-SA 4.0 (https://creativecommons.org/licenses/by-sa/4.0/)
%
%-------------------------------------------------------------------------------
% CONFIGURATIONS
%-------------------------------------------------------------------------------
% A4 paper size by default, use 'letterpaper' for US letter
\documentclass[11pt,a4paper,]{awesome-cv}

% Configure page margins with geometry
\usepackage{geometry}
\geometry{left=1.4cm, top=.8cm, right=1.4cm, bottom=1.8cm, footskip=.5cm}


% Specify the location of the included fonts
\fontdir[fonts/]

% Color for highlights
% Awesome Colors: awesome-emerald, awesome-skyblue, awesome-red, awesome-pink, awesome-orange
%                 awesome-nephritis, awesome-concrete, awesome-darknight

\definecolor{awesome}{HTML}{333333}

% Colors for text
% Uncomment if you would like to specify your own color
% \definecolor{darktext}{HTML}{414141}
% \definecolor{text}{HTML}{333333}
% \definecolor{graytext}{HTML}{5D5D5D}
% \definecolor{lighttext}{HTML}{999999}

% Set false if you don't want to highlight section with awesome color
\setbool{acvSectionColorHighlight}{true}

% If you would like to change the social information separator from a pipe (|) to something else
\renewcommand{\acvHeaderSocialSep}{\quad\textbar\quad}

\def\endfirstpage{\newpage}

%-------------------------------------------------------------------------------
%	PERSONAL INFORMATION
%	Comment any of the lines below if they are not required
%-------------------------------------------------------------------------------
% Available options: circle|rectangle,edge/noedge,left/right

\name{Juan David}{Leongómez Peña}

\position{Profesor Asociado}
\address{PhD University of Stirling · MSc University of Liverpool}

\mobile{(+57) 319 407 7102}
\email{\href{mailto:juanleongomez@gmail.com}{\nolinkurl{juanleongomez@gmail.com}}}
\homepage{jdleongomez.info}
\orcid{0000-0002-0092-6298}
\googlescholar{8Q0jKHsAAAAJ}

% \gitlab{gitlab-id}
% \stackoverflow{SO-id}{SO-name}
% \skype{skype-id}
% \reddit{reddit-id}


\usepackage{booktabs}

\providecommand{\tightlist}{%
	\setlength{\itemsep}{0pt}\setlength{\parskip}{0pt}}

%------------------------------------------------------------------------------


\usepackage{multicol}

% Pandoc CSL macros
\newlength{\cslhangindent}
\setlength{\cslhangindent}{1.5em}
\newlength{\csllabelwidth}
\setlength{\csllabelwidth}{2em}
\newenvironment{CSLReferences}[2] % #1 hanging-ident, #2 entry spacing
 {% don't indent paragraphs
  \setlength{\parindent}{0pt}
  % turn on hanging indent if param 1 is 1
  \ifodd #1 \everypar{\setlength{\hangindent}{\cslhangindent}}\ignorespaces\fi
  % set entry spacing
  \ifnum #2 > 0
  \setlength{\parskip}{#2\baselineskip}
  \fi
 }%
 {}
\usepackage{calc}
\newcommand{\CSLBlock}[1]{#1\hfill\break}
\newcommand{\CSLLeftMargin}[1]{\parbox[t]{\csllabelwidth}{\honortitlestyle{#1}}}
\newcommand{\CSLRightInline}[1]{\parbox[t]{\linewidth - \csllabelwidth}{\honordatestyle{#1}}}
\newcommand{\CSLIndent}[1]{\hspace{\cslhangindent}#1}

\begin{document}

% Print the header with above personal informations
% Give optional argument to change alignment(C: center, L: left, R: right)
\makecvheader

% Print the footer with 3 arguments(<left>, <center>, <right>)
% Leave any of these blank if they are not needed
% 2019-02-14 Chris Umphlett - add flexibility to the document name in footer, rather than have it be static Curriculum Vitae
\makecvfooter
  {13 de noviembre de 2023}
    {Juan David Leongómez Peña~~~·~~~Hoja de Vida Académica}
  {\thepage}


%-------------------------------------------------------------------------------
%	CV/RESUME CONTENT
%	Each section is imported separately, open each file in turn to modify content
%------------------------------------------------------------------------------



\vspace{4mm}
\begin{tcolorbox}[enhanced,
        on line, 
        boxsep=4pt, left=0pt,right=0pt,top=0pt,bottom=0pt,
        colframe=white,colback=black]
  
\color{white}
\begin{LARGE}\begin{center}
Documento 1. \textbf{Hoja de Vida}
\end{center}\end{LARGE}
\end{tcolorbox}

\hypertarget{resumen}{%
\section{Resumen}\label{resumen}}

\begin{footnotesize}

\textbf{Sobre mí:} Actualmente soy Profesor Asociado e Investigador en la Facultad de Psicología de la Universidad El Bosque, y lidero el grupo de investigación \href{https://investigaciones.unbosque.edu.co/codec}{\textit{\textbf{CODEC}: Ciencias Cognitivas y del Comportamiento}} (clasificación \textbf{\href{https://scienti.minciencias.gov.co/gruplac/jsp/visualiza/visualizagr.jsp?nro=00000000001446}{A1}}). 

Mi formación es \textit{sui generis} para el contexto colombiano. Después de formarme en música y pedagogía, obtuve una maestría en psicología evolutiva de la \href{https://www.liverpool.ac.uk/}{Universidad de Liverpool} (anteriormente ofrecida por la \href{https://www.liverpool.ac.uk/life-sciences/}{Escuela de Ciencias Biológicas}) y un doctorado en psicología de la \href{https://www.stir.ac.uk/}{Universidad de Stirling}, ambos en el Reino Unido. El doctorado lo realicé como parte del \textit{\href{https://www.stir.ac.uk/about/faculties/natural-sciences/our-research/research-groups/behaviour-and-evolution-research-group/}{Behaviour and Evolution Research Group}} del \href{https://www.stir.ac.uk/about/faculties/natural-sciences/psychology/}{Departamento de Psicología}, que en Stirling forma parte de la \href{https://www.stir.ac.uk/about/faculties/natural-sciences/departments/}{Facultad de Ciencias Naturales}. En consecuencia, mi perspectiva para la investigación del comportamiento humano está fuertemente influenciada por el enfoque y métodos de las ciencias naturales.

\textbf{Investigación:} Mis principales intereses de investigación se centran en el estudio evolutivo del comportamiento humano y su variabilidad cultural, y se sitúan en la interfaz entre biología y psicología, pero abarcando aspectos de etología, antropología y psicofísica. Como lo muestra mi producción académica, me interesa comprender los efectos sociales de distintas señales biológicas, y especialmente la percepción interpersonal a través de canales acústicos. En particular, mi investigación pionera sobre modulaciones vocales en distintos contextos sociales, ha recibido amplio reconocimiento internacional, como se documenta en \href{https://profiles.impactstory.org/u/0000-0002-0092-6298}{\textit{Impactstory}}, que registra las menciones de mi trabajo en medios de comunicación alrededor del mundo.

Aunque me he concentrado especialmente en el estudio de los procesos de elección de pareja y las señales vocales, también he trabajado señales faciales y morfológicas y sus efectos en la percepción de dominancia, tipicidad sexual y salud, entre otras. Así mismo, me interesan profundamente las bases de la musicalidad humana y sus diversas funciones sociales, así como los efectos de los niveles hormonales en el comportamiento social humano.

\textit{Edición especial sobre modulación de la voz}: Recientemente, propuse y lideré la edición del número temático en dos partes \href{https://royalsocietypublishing.org/toc/rstb/2021/376/1840}{\textit{Voice modulation: from origin and mechanism to social impact}} para la prestigiosa revista \href{https://royalsocietypublishing.org/journal/rstb}{\textit{Philosophical Transactions of the Royal Society B: Biological Sciences}}. Este número transdisciplinar en dos partes constituye el primer compendio de investigación sobre la modulación de la voz. Incluye 21 artículos que abarcan las diversas funciones de la modulación de la voz en la comunicación humana y no humana, desde la señalización no verbal de la motivación y la emoción, la exageración de rasgos biológicos como el tamaño corporal y la masculinidad, hasta la aparición del lenguaje, el canto y la musicalidad en todas las culturas.

\textbf{Ciencia abierta:} Apoyo firmemente la difusión de prácticas de ciencia abierta y el uso de software y lenguajes de programación de código abierto como herramientas para promover la transparencia y garantizar la reproducibilidad, así como para reducir las desigualdades nacionales e internacionales en la producción y el acceso al conocimiento. Por ello, he sido reconocido con un nombramiento como recomendador (editor) de la novedosa iniciativa \href{https://rr.peercommunityin.org/}{\textit{PCI Registered Reports}} (para más información sobre este modelo, ver \href{https://www.nature.com/articles/d41586-023-03342-6}{artículo en Nature} y la descripción de \href{https://rr.peercommunityin.org/about/about}{PCI}), y he publicado guías metodológicas en video a través del canal \href{https://www.youtube.com/@InvestigacionAbierta}{\textit{Investigación Abierta}}, así como aplicaciones web con fines educativos usando \textit{Shiny}. He participado activamente en la enseñanza y la transformación de módulos relacionados con la estadística, con un enfoque destacado en la ciencia abierta y la reproducibilidad, tanto con como sin programación.

\begin{cvskills}
  \cvskill
    {Intereses}
    {Modulación de la voz • Elección de pareja • Psicoacústica • Hormonas y comportamiento • Señales biológicas}
    
  \cvskill
    {}
    {Evolución y percepción de señales sociales}
    
  \cvskill
    {Métodos de Investigación}
    {\textit{Principalmente}: Diseños experimentales • Análisis acústico • Morfometría geométrica • Antropometría}

  \cvskill
    {Investigación Cuantitativa}
    {Modelado Estadístico • Modelos Multinivel • Metaanálisis • Inferencia Multimodelo • \textit{Machine learning}}
  
  \cvskill
    {Programación}
    {\href{https://www.r-project.org/}{\faRProject} avanzado: todo el procesamiento de datos, análisis, gráficos, tablas, páginas y \textit{apps} —e incluso esta HV—}

  \cvskill
    {Informes reproducibles}
    {Markdown/\href{https://rmarkdown.rstudio.com/}{R Markdown} (incluyendo código  \href{https://www.latex-project.org/}{{\fontfamily{cmr}\selectfont\LaTeX}} y \href{https://html.spec.whatwg.org/}{HTML}\faHtml5). Control de versiones con  \href{https://git-scm.com/}{Git} \faGit* y \href{https://github.com/JDLeongomez}{GitHub} \faGithub}

  \cvskill
    {Software}
    {\href{https://posit.co/products/open-source/rstudio/}{RStudio} •  \href{https://www.fon.hum.uva.nl/praat/}{Praat} • \href{https://www.audacityteam.org/}{Audacity} •  \href{https://inkscape.org/}{InkScape} • \href{https://www.zotero.org/}{Zotero} •  \href{https://www.jamovi.org/}{jamovi} • \href{https://jasp-stats.org/}{JASP} •  \href{https://obsproject.com/}{OBS Studio} • \href{https://code.visualstudio.com/}{VS Code} • \href{https://www.blender.org/}{Blender} • \href{https://www.gnu.org/}{GNU/Linux} \faLinux}

  \cvskill
    {Idiomas}
    {Inglés • Español}
\end{cvskills}

\end{footnotesize}

\hypertarget{formaciuxf3n-acaduxe9mica}{%
\section{Formación Académica}\label{formaciuxf3n-acaduxe9mica}}

\begin{cventries}
    \cventry{PhD - Psychology \textit{(\href{https://www.stir.ac.uk/about/faculties/natural-sciences/our-research/research-groups/behaviour-and-evolution-research-group/}{Behaviour and Evolution Research Group}, Faculty of Natural Sciences)}}{\href{https://www.stir.ac.uk/}{University of Stirling}}{Stirling, Reino Unido}{2014}{\begin{cvitems}
\item Tesis: \href{https://dspace.stir.ac.uk/handle/1893/21102}{\textbf{\textit{Contextual musicality: vocal modulation and its perception in human social interaction}}}
\item Supervisores: \href{https://www.scraigroberts.com/}{Prof. S. Craig Roberts}, y \href{https://scholar.google.com/citations?user=iDDoxVsAAAAJ}{Prof. Anthony C. Little}
\item Miembros del comité: \href{https://scholar.google.co.uk/citations?user=wxh9svQAAAAJ}{Prof. Phyllis C. Lee} (dissertation chair), y \href{https://scholar.google.com/citations?user=Qo23OGoAAAAJ}{Prof. Stuart Semple}
\end{cvitems}}
    \cventry{MSc in Evolutionary Psychology  \textit{(School of Biological Sciences)}}{\href{https://www.liverpool.ac.uk/}{University of Liverpool}}{Liverpool, Reino Unido}{2009}{\begin{cvitems}
\item Supervisor: \href{https://www.scraigroberts.com/}{Prof. S. Craig Roberts}
\item Mejor desempeño general en la maestría
\end{cvitems}}
    \cventry{Licenciatura en Pedagogía Musical}{\href{https://www.upn.edu.co/}{Universidad Pedagógica Nacional}}{Bogotá, Colombia}{2006}{\begin{cvitems}
\item Trabajo de Grado: 4.9/5.0
\end{cvitems}}
\end{cventries}

\hypertarget{formaciuxf3n-complementaria-relevante}{%
\subsection{Formación complementaria
relevante}\label{formaciuxf3n-complementaria-relevante}}

\begin{cventries}
    \cventry{Practical Machine Learning}{Johns Hopkins University}{Coursera (MOOC platform)}{2021}{\begin{cvitems}
\item Promedio: 97/100 (ver \href{https://www.coursera.org/account/accomplishments/verify/DC7ULMJ3CZWM}{certificado})
\end{cvitems}}
    \cventry{Statistical Programming in R and Linear Mixed Models}{University of Dundee}{Dundee, Reino Unido}{2012}{}\vspace{-4.0mm}
\end{cventries}

\hypertarget{experiencia-profesional}{%
\section{Experiencia Profesional}\label{experiencia-profesional}}

\begin{footnotesize}
En la actualidad, desempeño responsabilidades administrativas y de liderazgo tanto dentro como fuera de la Universidad El Bosque. Internamente, lidero el grupo de investigación \href{https://investigaciones.unbosque.edu.co/codec}{\textit{\textbf{CODEC}: Ciencias Cognitivas y del Comportamiento}} (clasificado como \href{https://scienti.minciencias.gov.co/gruplac/jsp/visualiza/visualizagr.jsp?nro=00000000001446}{A1}), y estoy a cargo de las convocatorias periódicas para el reconocimiento y medición de grupos de investigación, desarrollo tecnológico o innovación por parte de Minciencias. He organizado 9 cursos y soy miembro de dos comités de facultad. Externamente, tengo responsabilidades editoriales (detalladas a más adelante) y evalúo solicitudes de financiación para Minciencias, entre otras.
\end{footnotesize}

\hypertarget{principales-responsabilidades}{%
\subsection{Principales
Responsabilidades}\label{principales-responsabilidades}}

\begin{cventries}
    \cventry{Profesor Asociado}{\href{https://www.unbosque.edu.co/}{Universidad El Bosque}}{Bogotá, Colombia}{Ene. 2015 - Actualmente}{\begin{cvitems}
\item Investigador en \href{https://jdleongomez.info/es/team/}{\textit{\textbf{EvoCo}: Laboratorio de Evolución y Comportamiento Humano}}
\item Líder del grupo de investigación \href{https://investigaciones.unbosque.edu.co/codec}{\textit{\textbf{CODEC}: Ciencias Cognitivas y del Comportamiento}} (desde 2016)
\item \href{https://asesores-psic.netlify.app/}{Asesor metodológico y estadístico} para proyectos de investigación de posgrado y profesores de la Facultad
\item Supervisión de una variedad de proyectos de investigación de pregrado asociados con psicología y biología
\item Miembro del Comité de Investigación de la Facultad de Psicología
\item Miembro del Comité Asesor de Asuntos Éticos de la Facultad de Psicología
\item Organizador de la serie de conferencias semanales Café Nerd de la Facultad de Psicología (2016-2017)
\item Estancia posdoctoral (2018-2019)
\item Co-supervisión de estudiantes de doctorado: \href{https://www.researchgate.net/profile/Milena-Vasquez-Amezquita}{Milena Vásquez-Amézquita} (PhD en Neurociencia, Universidad de Valencia, España - 2015-2018). Francisco Javier Flores  (Professional Doctorate in Counselling Psychology, University of East London, Reino Unido – 2016-2018)
\end{cvitems}}
    \cventry{Profesor Catedrático}{\href{https://www.unisabana.edu.co/}{Universidad de La Sabana}}{Chia, Colombia}{Ene. 2015 - Dic. 2016}{\begin{cvitems}
\item Docencia y supervisión de proyectos de investigación
\end{cvitems}}
    \cventry{\textit{Recognised teacher in Psychology}}{\href{https://www.stir.ac.uk/}{University of Stirling}}{Stirling, Reino Unido}{2011 - 2014}{\begin{cvitems}
\item Supervisión de un proyecto de investigación de maestría (Evolutionary Psychology MSc)
\item Calificación y asesoramiento estadístico a estudiantes de maestría
\end{cvitems}}
    \cventry{Profesor Auxiliar}{\href{https://www.upn.edu.co/}{Universidad Pedagógica Nacional}}{Bogotá, Colombia}{2010}{\begin{cvitems}
\item Miembro del Comité de Investigación
\item Supervisión de proyectos de investigación
\end{cvitems}}
\end{cventries}

\hypertarget{participaciuxf3n-en-proyectos-de-investigaciuxf3n}{%
\subsection{Participación en Proyectos de
investigación}\label{participaciuxf3n-en-proyectos-de-investigaciuxf3n}}

\begin{cventries}
    \cventry{Proyecto: \textit{Efecto del control de los recursos real y simulado sobre las preferencias de mujeres andrófilas por la masculinidad en rostros de hombres: un estudio experimental usando rastreo ocular}}{}{}{Por definir}{\begin{cvitems}
\item \textbf{Entidad financiadora:} \href{https://www.unbosque.edu.co/}{Universidad El Bosque}
\item \textbf{Entidad ejecutora:} \href{https://www.unbosque.edu.co/}{Universidad El Bosque}
\item \textbf{Investigadores principales:} \href{https://scholar.google.es/citations?user=XgNEpfgAAAAJ}{Milena Vásquez-Amézquita}
\item \textbf{Cargo:} Investigador
\end{cvitems}}
    \cventry{Proyecto: \textit{Disgust Perception through Sounds and Sights}}{}{}{Desde Oct. 2021}{\begin{cvitems}
\item \textbf{Entidad financiadora:} Proyecto solidario
\item \textbf{Investigadores principales:} \href{https://lsa.umich.edu/psych/people/faculty/joshack.html}{Joshua M. Ackerman}  (global) y \href{https://jdleongomez.info/es/}{Juan David Leongómez} (Colombia)
\item \textbf{Cargo:} Investigador principal en UEB
\end{cvitems}}
    \cventry{Proyecto: \textit{Señales de infidelidad y sociosexualidad en rostros y voces}}{}{}{Desde Jul. 2019}{\begin{cvitems}
\item \textbf{Entidad financiadora:} \href{https://www.unbosque.edu.co/}{Universidad El Bosque}
\item \textbf{Entidad ejecutora:} \href{https://www.unbosque.edu.co/}{Universidad El Bosque}
\item \textbf{Investigadores principales:} \href{https://jdleongomez.info/es/}{Juan David Leongómez}
\item \textbf{Cargo:} Investigador Principal
\end{cvitems}}
    \cventry{Proyecto: \textit{Modulación vocal}}{}{}{Desde Ene. 2018}{\begin{cvitems}
\item \textbf{Entidad financiadora:} Proyecto solidario
\item \textbf{Investigadores principales:} \href{https://jdleongomez.info/es/}{Juan David Leongómez} y \href{https://scholar.google.com/citations?user=sTCqMrwAAAAJ}{Katarzyna Pisanski}
\item \textbf{Cargo:} Editor e Investigador Principal
\end{cvitems}}
    \cventry{Proyecto: \textit{Señales perceptibles de salud física y mental en rostros, voces y olores corporales, y su relación con niveles hormonales}}{}{}{Desde Ene. 2018}{\begin{cvitems}
\item \textbf{Entidad financiadora:} \href{https://www.unbosque.edu.co/}{Universidad El Bosque}
\item \textbf{Entidad ejecutora:} \href{https://www.unbosque.edu.co/}{Universidad El Bosque}
\item \textbf{Investigadores principales:} \href{https://jdleongomez.info/es/}{Juan David Leongómez}
\item \textbf{Cargo:} Investigador Principal
\end{cvitems}}
    \cventry{Proyecto: \textit{Efecto de señales estáticas evolutivamente relevantes (sexo, dominancia y atractivo) en el procesamiento cortical de rostros humanos}}{}{}{Desde Ene. 2017}{\begin{cvitems}
\item \textbf{Entidad financiadora:} \href{https://www.unbosque.edu.co/}{Universidad El Bosque}
\item \textbf{Entidad ejecutora:} \href{https://www.unbosque.edu.co/}{Universidad El Bosque}
\item \textbf{Investigadores principales:} \href{https://scholar.google.com/citations?user=1X-h1-YAAAAJ}{Eugenio Valderrama }
\item \textbf{Cargo:} Investigador
\end{cvitems}}
    \cventry{Proyecto: \textit{Evolución de la musicalidad}}{}{}{Jun. 2018 - Jun. 2021}{\begin{cvitems}
\item \textbf{Entidad financiadora:} Proyecto solidario
\item \textbf{Investigadores principales:} \href{https://jdleongomez.info/es/}{Juan David Leongómez} y \href{https://scholar.google.com/citations?user=jPPF5y0AAAAJ}{S. Craig Roberts}
\item \textbf{Cargo:} Investigador Principal
\end{cvitems}}
    \cventry{Proyecto: \textit{Predictores acústicos de fuerza física en poblaciones africanas}}{}{}{Jun. 2018 - May. 2021}{\begin{cvitems}
\item \textbf{Entidad financiadora:} Proyecto solidario
\item \textbf{Investigadores principales:} \href{https://scholar.google.com/citations?user=ZtzwhVsAAAAJ}{Karel Kleisner}, \href{https://jdleongomez.info/es/}{Juan David Leongómez} y \href{https://scholar.google.com/citations?user=sTCqMrwAAAAJ}{Katarzyna Pisanski}
\item \textbf{Cargo:} Co-investigador Principal
\end{cvitems}}
    \cventry{Proyecto: \textit{Does Oosterhof and Todorov’s valence-dominance model of social perception of faces generalize across world regions?}}{}{}{Mar. 2018 - Ene. 2021}{\begin{cvitems}
\item \textbf{Entidad financiadora:} \href{https://psysciacc.org/}{Psychological Science Accelerator}
\item \textbf{Investigadores principales:} \href{https://scholar.google.com/citations?user=N6qw59YAAAAJ}{Benedict C. Jones} y \href{https://scholar.google.com/citations?user=oLHcVYUAAAAJ}{Lisa DeBruine}
\item \textbf{Cargo:} Investigador Principal Universidad El Bosque
\end{cvitems}}
    \cventry{Proyecto: \textit{Señales perceptibles de salud física y mental en rostros, voces y olores corporales, y su relación con niveles hormonales}}{}{}{Ene. 2018 - Ene. 2019}{\begin{cvitems}
\item \textbf{Entidad financiadora:} \href{https://minciencias.gov.co/}{Minciencias}
\item \textbf{Entidad ejecutora:} \href{https://www.unbosque.edu.co/}{Universidad El Bosque}
\item \textbf{Investigadores principales:} \href{https://jdleongomez.info/es/}{Juan David Leongómez}
\item \textbf{Cargo:} Investigador Principal
\end{cvitems}}
    \cventry{Proyecto: \textit{Efectos de los niveles hormonales, masculinidad y feminidad, en la discriminación tonal en hombres y mujeres}}{}{}{Ene. 2016 - Jun. 2019}{\begin{cvitems}
\item \textbf{Entidad financiadora:} \href{https://www.unbosque.edu.co/}{Universidad El Bosque}
\item \textbf{Entidad ejecutora:} \href{https://www.unbosque.edu.co/}{Universidad El Bosque}
\item \textbf{Investigadores principales:} \href{https://jdleongomez.info/es/}{Juan David Leongómez}
\item \textbf{Cargo:} Investigador Principal
\end{cvitems}}
    \cventry{Proyecto: \textit{Transmission of information in language and music}}{}{}{Oct. 2010 - Sep. 2014}{\begin{cvitems}
\item \textbf{Entidad financiadora:} \href{https://minciencias.gov.co/}{Minciencias}
\item \textbf{Entidad ejecutora:} \href{https://www.stir.ac.uk/}{University of Stirling}
\item \textbf{Investigadores principales:} \href{https://jdleongomez.info/es/}{Juan David Leongómez}
\item \textbf{Cargo:} Investigador Principal
\end{cvitems}}
\end{cventries}

\hypertarget{experiencia-docente}{%
\subsection{Experiencia Docente}\label{experiencia-docente}}

\begin{footnotesize}
A partir de mi trabajo en la Universidad de Stirling (Escocia, Reino Unido), mi enfoque principal ha sido la enseñanza de cursos relacionados con la estadística y metodos cuantitativos. Tanto en la Universidad El Bosque, donde impartí clases a nivel de maestría, como en la Universidad de La Sabana, donde trabajé a nivel de pregrado, rediseñé los programas de aprendizaje de estadística para el uso de software especializado, con énfasis en la aplicabilidad y credibilidad de los resultados.

Desde el año 2020, lideré una iniciativa en la Universidad El Bosque para transitar de la dependencia de software propietario, costoso y anticuado como SPSS, a soluciones amigables, libres y de código abierto, siguiendo los principios de la ciencia abierta y centrándonos en el modelado estadístico, la inferencia y la reproducibilidad. Este cambio ha implicado la implementación de \textit{jamovi} en cursos de pregrado y posgrado en la Facultad de Psicología, así como la realización de seminarios sobre R para el cuerpo docente y el co-diseño de un módulo opcional de R para estudiantes de pregrado.

Es relevante destacar que entre 2015 y 2019, mi enfoque principal fue la docencia. A partir de 2020, aunque ya no tengo a mi cargo cursos, he asumido mayores responsabilidades en investigación, supervisión de trabajos de grado y asesorías metodológicas y estadísticas para proyectos de investigación, tanto en el ámbito docente como de posgrado.
\end{footnotesize}

\begin{cventries}
    \cventry{Profesor Asociado}{\href{https://www.unbosque.edu.co/}{Universidad El Bosque}}{Bogotá, Colombia}{2015 - 2019}{\begin{cvitems}
\item Métodos cuantitativos II (Maestría en Psicología) (2017-2019)
\item Métodos cuantitativos I (Maestría en Psicología) (2017)
\item Taller Proyecto de Grado I (2018)
\item Fuentes y estilos de documentación en psicología (2015)
\end{cvitems}}
    \cventry{Profesor Catedrático}{\href{https://www.unisabana.edu.co/}{Universidad de La Sabana}}{Chia, Colombia}{2015 - 2016}{\begin{cvitems}
\item Evolución y desarrollo de la comunicación vocal: cantos, modas y lenguaje (2016)
\item Estadística inferencial (2015 - 2016)
\item Estadística descriptiva (2015 - 2016)
\end{cvitems}}
    \cventry{\textit{Recognised teacher in Psychology}}{\href{https://www.stir.ac.uk/}{University of Stirling}}{Stirling, Reino Unido}{2012 - 2014}{\begin{cvitems}
\item Comportamiento animal (cátedra de comunicación vocal) (2012)
\item Métodos cuantitativos (Maestría en psicología: varias conferencias, supervisión práctica, enseñanza individualizada) (2012-2014)
\item Módulo de Cognición (liderando proyectos de investigación en psicoacústica) (2012-2014)
\end{cvitems}}
    \cventry{Profesor Auxiliar}{\href{https://www.upn.edu.co/}{Universidad Pedagógica Nacional}}{Bogotá, Colombia}{2010}{\begin{cvitems}
\item Proyecto de investigación I (2010)
\item Laboratorio de investigación II (2010)
\end{cvitems}}
\end{cventries}

\blacktriangleright\blacktriangleright\blacktriangleright\space \textbf{Evaluaciones de estudiantes}
(Universidad El Bosque)

\begin{footnotesize}
A continuación, se presenta un resumen de las evaluaciones de estudiantes de los cursos que he dictado en la Universidad El Bosque, así como de estudiantes que han realizado trabajos de grado bajo mi supervisión. Todos los valores corresponden a la media de las evaluaciones obtenidas durante diferentes semestres.
\end{footnotesize}

\begin{cventries}
    \cventry{Los valores corresponden al porcentaje medio de estudiantes que calificaron cada aspecto como excelente \newline (las opciones son \textbf{Muy malo}, \textbf{Malo}, \textbf{Aceptable}, \textbf{Bueno} y \textbf{Excelente})}{Dirección trabajos de grado \textbf{(promedio 2020-2 a 2023-2)}}{Universidad El Bosque}{2020-2023}{\begin{cvitems}
\item Guía a la investigación: \textbf{100\%}
\item Compromiso con la investigación: \textbf{91.1\%}
\item Conocimiento temático: \textbf{91.1\%}
\item Conocimiento metodológico: \textbf{89.3\%}
\item Orientación para la búsqueda de información: \textbf{91.7\%}
\item Coherencia en el proceso: \textbf{89.9\%}
\item Manejo de normas APA: \textbf{86.4\%}
\item Relaciones interpersonales: \textbf{93.5\%}
\item Habilidad en solución de problemas: \textbf{85.8\%}
\item Puntualidad en los encuentros: \textbf{86.9\%}
\item Puntualidad en la retroalimentación de los avances del proyecto: \textbf{84.5\%}
\item Claridad en la retroalimentación: \textbf{91.7\%}
\item Diligenciamiento del formato de control de cada una de las asesorías: \textbf{82.1\%}
\item Asignación oportuna de notas: \textbf{86.3\%}
\item Conocimiento de las fechas y los procesos administrativos relacionados con la investigación: \textbf{81.5\%}
\item Uso del aula virtual del Centro de Investigaciones como herramienta de apoyo: \textbf{76.9\%}
\item \textbf{Nota media del papel como director} (sobre 5.0)\textbf{: 4.87}
\end{cvitems}}
    \cventry{Los valores corresponden al porcentaje medio de estudiantes que calificaron cada logro como \textbf{Alto} \newline (las opciones son \textbf{Bajo}, \textbf{medio} y \textbf{Alto})}{Taller Proyecto de Grado I}{Universidad El Bosque}{2018}{\begin{cvitems}
\item Ser capaz de leer sobre la materia de esta área de forma más crítica: \textbf{90.9\%}
\item Corregir ideas y conceptos inexactos sobre este campo de la psicología: \textbf{90.9\%}
\item Interés por la investigación sobre el área objeto de estudio: \textbf{90.9\%}
\item Nuevos conceptos e investigaciones: \textbf{ 81.8\%}
\item Aplicación de conocimientos en el contexto psicológico: \textbf{81.8\%}
\end{cvitems}}
    \cventry{Los valores corresponden al porcentaje medio de estudiantes que calificaron cada logro como \textbf{Alto} \newline (las opciones son \textbf{Bajo}, \textbf{medio} y \textbf{Alto})}{Fuentes y estilos de documentación en psicología}{Universidad El Bosque}{2015}{\begin{cvitems}
\item Ser capaz de leer sobre la temática de esta área de forma más crítica: \textbf{79.5\%}
\item Corregir ideas y conceptos inexactos sobre este campo de la psicología: \textbf{70.5\%}
\item Interés por la investigación sobre el área objeto de estudio: \textbf{77.3\%}
\item Nuevos conceptos e investigaciones: \textbf{72.7\%}
\item Aplicación de los conocimientos en el contexto psicológico: \textbf{81.8\%}
\end{cvitems}}
    \cventry{Los valores corresponden a la puntuación media \textbf{sobre 5.0}}{Métodos cuantitativos II (Maestría en Psicología)}{Universidad El Bosque}{2017-2019}{\begin{cvitems}
\item Relevancia de los contenidos para el objetivo del programa: \textbf{4.75}
\item Material aportado para apoyar el desarrollo del módulo: \textbf{4.67}
\item Recursos didácticos utilizados: \textbf{4.64}
\item Utilidad práctica de los contenidos: \textbf{4.63}
\item Contribución de los contenidos a la vida profesional: \textbf{4.63}
\item Cumplimiento del programa: \textbf{4.76}
\item Diseño de las sesiones: \textbf{4.69}
\item \textbf{ Media Aspectos Académicos y Metodológicos: 4.69}
\item Manejo conceptual claro y completo de los temas: \textbf{4.65}
\item Claridad en sus exposiciones: \textbf{4.49}
\item Terminología utilizada: \textbf{4.74}
\item Uso de estrategias que desarrollan el pensamiento crítico-constructivo: \textbf{4.60}
\item Uso de estrategias que facilitan la comprensión de conceptos: \textbf{4.61}
\item Uso de estrategias que facilitan la aplicación de los conceptos abordados en la asignatura: \textbf{4.63}
\item Disposición para resolver dudas: \textbf{4.71}
\item Fomento de la participación en clase: \textbf{4.70}
\item \textbf{ Media Condiciones Pedagógicas y Didácticas: 4.65}
\item Volumen de lecturas adecuado y objetivo: \textbf{4.70}
\item Fomento del interés del alumno por aprender: \textbf{4.72}
\item Retroalimentación de las actividades realizadas (talleres, seminarios, análisis de casos): \textbf{4.57}
\item \textbf{Media Facilitadores del Aprendizaje: 4.71}
\end{cvitems}}
\end{cventries}

\blacktriangleright\blacktriangleright\blacktriangleright\space \textbf{Comentarios de estudiantes}
(Universidad El Bosque)

\begin{footnotesize}
A continuación figuran algunos comentarios anónimos incluidos en las evaluaciones realizadas por estudiantes de los cursos que he dictado en la Universidad El Bosque, así como de estudiantes que han realizado trabajos de grado bajo mi supervisión.
\end{footnotesize}

\begin{cventries}
    \cventry{Comentarios incluyen:}{Dirección trabajos de grado}{Universidad El Bosque}{2020-2023}{\begin{cvitems}
\item \textit{'Me encantó la supervisión por parte de Juan David, fue muy claro en su enseñanza, y aprendí muchas cosas importantes para mi vida profesional'}
\item \textit{'Excelente todo el procesos investigativo'}
\item \textit{'Muy buen acompañamiento en el proceso de formación'}
\end{cvitems}}
    \cventry{Comentarios incluyen:}{Taller Proyecto de Grado I}{Universidad El Bosque}{2018}{\begin{cvitems}
\item \textit{'Excelente profesor, buena metodología y siempre disponible para las tutorías'}
\item \textit{'Excelente actitud del profesor durante la clase y en las tutorías'}
\item \textit{'El profesor sabe mucho y su metodología es adecuada'}
\end{cvitems}}
    \cventry{Comentarios incluyen:}{Fuentes y estilos de documentación en psicología}{Universidad El Bosque}{2015}{\begin{cvitems}
\item \textit{'El profesor Juan siempre venía preparado a las clases y nos mantenía informados de todo, los temas eran muy interesantes y aportaban mucho a mi aprendizaje; si tenía alguna dificultad o inquietud se preocupaba por resolverla'}
\item \textit{'¡Excelente profesor! Se desvive para que aprendamos y veamos la importancia de la investigación en psicología'}
\item \textit{'Es un buen profesor que se esfuerza para que aprendamos y seamos personas con sentido crítico y que explica mediante metodologías que facilitan la comprensión de los temas'}
\end{cvitems}}
    \cventry{Comentarios incluyen:}{Métodos cuantitativos II (Maestría en Psicología)}{Universidad El Bosque}{2017-2019}{\begin{cvitems}
\item \textit{'Juan David tiene una muy buena forma de transmitir los conocimientos, sobre todo para las personas que no somos muy duchos en temas cuantitativos'}
\item \textit{'El curso fue excelente, el profesor brindó un espacio de aprendizaje muy valioso, no sólo porque los temas eran muy relevantes y a nivel de maestría, sino porque motivó nuestro pensamiento crítico, nos dio herramientas poco convencionales y en general fue flexible a las peticiones del grupo'}
\item \textit{'Excelente profesional y ser humano'}
\item \textit{'Considero que el aporte del profesor fue excelente, me permitió obtener las competencias necesarias para el campo investigativo y estadístico'}
\end{cvitems}}
\end{cventries}

\hypertarget{financiaciuxf3n}{%
\subsection{Financiación}\label{financiaciuxf3n}}

\begin{footnotesize}
La financiación para la investigación científica en Colombia es escasa\footnote{En 2015, Colombia \href{https://www.science.org/content/article/researchers-thought-peace-colombia-would-mean-more-science-funding-they-were-wrong}{gastó apenas el 0,24\% de su producto interno bruto en I+D}, y desde entonces, la situación ha empeorado. En 2018, el \href{https://www-infobae-com.translate.goog/america/colombia/2021/09/03/congresistas-piden-incrementar-el-presupuesto-de-minciencia-para-2022/?_x_tr_sl=es&_x_tr_tl=en}{presupuesto de Minciencias} era de US\$75.000.000, y para 2021 solo rondaba los US\$67.000.000.}, especialmente para la investigación básica; del limitado número de becas disponibles, gran parte son para investigación en salud o proyectos de conservación, así como para abordar retos específicos del posconflicto, lo que significa que mis opciones para obtener financiación son restringidas. Además, incluso la financiación interna de la Universidad El Bosque se ha desplazado recientemente hacia ciertas áreas clave, dirigidas en todos los casos a la investigación aplicada y en línea con las áreas priorizadas establecidas por Minciencias.

A pesar de ello, he superado sistemáticamente a mis colegas y nunca he dejado de obtener financiación para ninguna de las solicitudes que he presentado desde 2015. De hecho, desde el inicio de mi doctorado he recibido más de US\$200.000 en financiación a través de subvenciones y becas externas, incluyendo fondos de Minciencias.
\end{footnotesize}

\begin{cventries}
    \cventry{XIV \href{https://investigaciones.unbosque.edu.co/s/Resultados-finales-Convocatoria-Interna-2024.pdf}{Convocatoria Interna de Investigaciones 2024}}{\href{https://www.unbosque.edu.co/}{Universidad El Bosque}}{Bogota, Colombia}{}{\begin{cvitems}
\item Efecto del control de los recursos real y simulado sobre las preferencias de mujeres andrófilas por la masculinidad en rostros de hombres: un estudio experimental usando rastreo ocular
\item COP\$90.000.000
\item Investigadora Pirncipal: \href{https://scholar.google.es/citations?user=XgNEpfgAAAAJ}{Milena Vásquez-Amézquita}
\end{cvitems}}
    \cventry{\href{https://minciencias.gov.co/convocatorias/investigacion/convocatoria-programa-estancias-postdoctorales-beneficiarios-colciencias}{Convocatoria programa de Estancias Postdoctorales beneficiarios Colciencias 2017}}{\href{https://minciencias.gov.co/}{Minciencias}}{Bogota, Colombia}{Ene. 2018 - Ene. 2019}{\begin{cvitems}
\item Proyecto: Señales perceptibles de salud física y mental en rostros, voces y olores corporales, y su relación con niveles hormonales
\item COP\$84.000.000
\item Investigador Pirncipal: \href{https://jdleongomez.info/es/}{Juan David Leongómez}
\end{cvitems}}
    \cventry{IX \href{https://www.unbosque.edu.co/investigaciones/convocatorias-investigacion}{Convocatoria Interna para la Financiación de Proyectos de Investigación}, 2017}{\href{https://www.unbosque.edu.co/}{Universidad El Bosque}}{Bogota, Colombia}{Ene. 2018 - Dic. 2021}{\begin{cvitems}
\item Proyecto: Señales perceptibles de salud física y mental en rostros, voces y olores corporales, y su relación con niveles hormonales
\item COP\$136.586.537
\item Investigador Pirncipal: \href{https://jdleongomez.info/es/}{Juan David Leongómez}
\end{cvitems}}
    \cventry{VIII \href{https://www.unbosque.edu.co/investigaciones/convocatorias-investigacion}{Convocatoria Interna para la Financiación de Proyectos de Investigación}, 2016}{\href{https://www.unbosque.edu.co/}{Universidad El Bosque}}{Bogota, Colombia}{Ene. 2017 - Dic. 2020}{\begin{cvitems}
\item Proyecto: Efecto de señales estáticas evolutivamente relevantes (sexo, dominancia y atractivo) en el procesamiento cortical de rostros humanos
\item COP\$80.000.000
\item Investigador Pirncipal: \href{https://scholar.google.com/citations?user=1X-h1-YAAAAJ}{Eugenio Valderrama }
\end{cvitems}}
    \cventry{VII \href{https://www.unbosque.edu.co/investigaciones/convocatorias-investigacion}{Convocatoria Interna para la Financiación de Proyectos de Investigación}, 2015}{\href{https://www.unbosque.edu.co/}{Universidad El Bosque}}{Bogota, Colombia}{Ene. 2016 - Dic. 2019}{\begin{cvitems}
\item Proyecto: Efectos de los niveles hormonales, masculinidad y feminidad, en la discriminación tonal en hombres y mujeres
\item COP\$13.000.000
\item Investigador Pirncipal: \href{https://jdleongomez.info/es/}{Juan David Leongómez}
\end{cvitems}}
\end{cventries}

\hypertarget{becas-premios-y-honores}{%
\subsection{Becas, Premios y Honores}\label{becas-premios-y-honores}}

\begin{cventries}
    \cventry{IX Convocatoria de Estímulos a la Excelencia}{\href{https://www.unbosque.edu.co/}{Universidad El Bosque}}{Bogotá, Colombia}{Dic. 2022}{\begin{cvitems}
\item COP\$10.000.000
\end{cvitems}}
    \cventry{Economics Prize}{\href{https://improbable.com/ig/about-the-ig-nobel-prizes/}{Ig Nobel Prize}}{Cambridge, MA, EE.UU.}{Sep. 2020}{\begin{cvitems}
\item Por ‘tratar de cuantificar la relación entre la desigualdad de ingresos nacionales en diferentes países y la cantidad promedio de besos boca a boca’ (\href{https://doi.org/10.1038/s41598-019-43267-7}{Watkins, et al., 2019})
\end{cvitems}}
    \cventry{VIII Convocatoria de Estímulos a la Excelencia}{\href{https://www.unbosque.edu.co/}{Universidad El Bosque}}{Bogotá, Colombia}{Dic. 2019}{\begin{cvitems}
\item COP\$7.000.000
\end{cvitems}}
    \cventry{VII Convocatoria de Estímulos a la Excelencia}{\href{https://www.unbosque.edu.co/}{Universidad El Bosque}}{Bogotá, Colombia}{Dic. 2018}{\begin{cvitems}
\item COP\$5.000.000
\end{cvitems}}
    \cventry{VI Convocatoria de Estímulos a la Excelencia}{\href{https://www.unbosque.edu.co/}{Universidad El Bosque}}{Bogotá, Colombia}{Dic. 2017}{\begin{cvitems}
\item COP\$5.000.000
\end{cvitems}}
    \cventry{Grindley Grants}{\href{https://eps.ac.uk/}{Experimental Psychology Society}}{Canterbury, Reino Unido}{Sep. 2014}{}\vspace{-4.0mm}
    \cventry{Becas Francisco José de Caldas}{\href{https://minciencias.gov.co/}{Minciencias}}{Bogotá, Colombia}{Oct. 2010 - Oct. 2014}{}\vspace{-4.0mm}
    \cventry{Annual Prize in Evolutionary Psychology}{\href{https://www.liverpool.ac.uk/life-sciences/}{School of Life Sciences} – University of Liverpool}{Liverpool, Reino Unido}{Oct. 2009}{\begin{cvitems}
\item Mejor desempeño general en la maestría
\end{cvitems}}
    \cventry{University of Liverpool International Scholarship}{\href{https://www.liverpool.ac.uk/}{University of Liverpool}}{Liverpool, Reino Unido}{Sep. 2008 - Sep. 2009}{}\vspace{-4.0mm}
    \cventry{Programa Crédito-Beca}{\href{https://www.colfuturo.org/}{Colfuturo}}{Bogotá, Colombia}{Sep. 2008 - Sep. 2009}{}\vspace{-4.0mm}
\end{cventries}

\hypertarget{cursos-y-congresos}{%
\subsection{Cursos y congresos}\label{cursos-y-congresos}}

\begin{cventries}
    \cventry{Presidente del Comité Científico}{\href{https://www.youtube.com/playlist?list=PLI4QwBEXHFJKVhSBqaof6gJdKCi3CN2UM}{CIVN2020 - Congreso Internacional de Neurociencias: Cerebro y Comportamiento en Tiempos de COVID-19}}{Universidad El Bosque y Universidad de los Andes}{Nov. 25 ‑ 28, 2020}{\begin{cvitems}
\item \href{http://doi.org/10.17605/OSF.IO/5BWNX}{Memorias}
\item \href{https://www.youtube.com/@onlineinternationalcongres6942}{Canal de YouTube} (Todo el congreso disponible)
\end{cvitems}}
    \cventry{Miembro del Comité Científico}{\href{https://www.isep.es/congreso2020/}{1er Congreso Internacional Virtual ISEP}}{ISEP}{May. 5 - 7, 2020}{\begin{cvitems}
\item \href{https://www.isep.es/congreso2020/\#comite}{Comité Científico}
\end{cvitems}}
\end{cventries}

\blacktriangleright\blacktriangleright\blacktriangleright\space \textbf{Conferencias, Pósters y Talleres}

\begingroup
\footnotesize
\setlength{\parindent}{-0.5in}
\setlength{\leftskip}{0.5in}

\textbf{Leongómez, J. D.} (2023, agosto). \emph{Potenciando tus
conclusiones: Explorando la vital importancia del poder estadístico y el
tamaño de muestra en la inferencia confiable}. Presentación invitada al
XXIX Congreso Institucional de Investigaciones (Universidad El Bosque),
Bogotá, Colombia.
\url{https://investigaciones.unbosque.edu.co/s/D-00730__ViceInvestigaciones__-agenda-Congreso-Institucional-links-1.pdf}

\textbf{Leongómez, J. D.} (2023, julio). \emph{From data to knowledge:
Conducting meta-analyses of correlations and t-tests}. Taller invitado
presentado en el Summer Institute 2023 de of the International Society
for Human Ethology (ISHE), Universidade Federal de Pernambuco, Recife,
Brasil. \url{https://ishe.org/wp-content/uploads/2023/07/BoA_v06_km.pdf}
(ver
\href{https://jdleongomez.info/es/files/Cert_ISHE_2023.pdf}{certificado})

\textbf{Leongómez, J. D.} (2023, julio). \emph{Ensuring reliable
results: Power analysis and sample size estimation in behavioural
research}. Taller invitado presentado en el Summer Institute 2023 de of
the International Society for Human Ethology (ISHE), Universidade
Federal de Pernambuco, Recife, Brasil.
\url{https://ishe.org/wp-content/uploads/2023/07/BoA_v06_km.pdf} (ver
\href{https://jdleongomez.info/es/files/Cert_ISHE_2023.pdf}{certificado})

\textbf{Leongómez, J. D.}, Havlíček, J., \& Roberts, S.C. (2022,
octubre). \emph{Musicality in human vocal communication}. Artículo
presentado como parte del simposio The Evolution of Human Musicality en
el 4th Brazilian Meeting on Evolution of Human Behavior (Universidad de
São Paulo), São Paulo, Brasil. \url{https://youtu.be/gK_rl0r4nfs?t=9662}

\textbf{Leongómez, J. D.} (2022, octubre). \emph{Sesión presencial de
jamovi}. Taller para el Plan de formación en investigación de la
Vicerrectoría de Investigaciones (Universidad El Bosque), Bogotá,
Colombia.

\textbf{Leongómez, J. D.} (2022, septiembre). \emph{Introducción a
jamovi: Alternativa moderna, gratuita y de código abierto para la
realización de análisis estadísticos}. Presentación organizada como
parte del Plan de formación en investigación de la Vicerrectoría de
Investigaciones (Universidad El Bosque), Bogotá, Colombia.
\url{https://investigaciones.unbosque.edu.co/eventos/cuestioneseticas-jjnlm-92bkj-n249w-hbnhl-6tpj4-pm22x-8m3yh-645tt-gw59h-xwc6x-d9sgw-6fzel-hm6dd-2wrls-569m2-8zns8-rg363}

\textbf{Leongómez, J. D.} (2022, agosto). \emph{Más allá del Open
Access: Necesidad de una Ciencia Abierta integral para enfrentar las
implicaciones de la naturalización de malas prácticas y la crisis de la
replicación}. Presentación organizada como parte del Plan de formación
en investigación de la Vicerrectoría de Investigaciones (Universidad El
Bosque), Bogotá, Colombia.
\url{https://investigaciones.unbosque.edu.co/eventos/cuestioneseticas-jjnlm-92bkj-n249w-hbnhl-6tpj4-pm22x-8m3yh-645tt-gw59h-xwc6x-d9sgw-6fzel-hm6dd-2wrls-569m2-8zns8}

\textbf{Leongómez, J. D.} (2022, abril). \emph{¿Qué tan confiable es la
literatura científica? Ciencia abierta como estrategia para enfrentar
las implicaciones de la naturalización de malas prácticas y la crisis de
la replicación}. Presentación organizada por la Unidad de
Investigaciones de la Facultad de Psicología (Universidad El Bosque),
Bogotá, Colombia. \url{https://youtu.be/Fl98MoBqSaI}

\textbf{Leongómez, J. D.} (2020, noviembre). \emph{¿Cuántos
participantes necesito? Análisis de poder estadístico y cálculo de
tamaño de muestra en R}. Taller para el Congreso Internacional de
Neurociencias: Cerebro y Comportamiento en Tiempos de COVID-19
(Universidad El Bosque, Universidad de los Andes), Bogotá, Colombia.
\url{https://youtu.be/tVB9sh5ZFH0}

\textbf{Leongómez, J. D.} \& Sánchez, O. R. (2019, septiembre).
\emph{Asociación entre altura y circunferencia de cintura con salud en
poblaciones rurales indígenas y urbanas en América Latina}. Artículo
presentado en el XXV Congreso Institucional de Investigaciones
(Universidad El Bosque), Bogotá, Colombia.
\url{https://static1.squarespace.com/static/5fb667be752a000eaf4b5869/t/60454941f7f209159575a773/1615153475374/Memorias+-XXV+Congreso+2019.pdf}

\textbf{Leongómez, J. D.}, Sánchez, O.R., Vásquez-Amézquita, M.,
Valderrama, E., Castellanos-Chacón, A., Morales-Sánchez, L., Nieto, J.,
\& González-Santoyo, I. (2019, agosto). \emph{Self-reported Health is
Related to Body Height and Waist Circumference in Rural Indigenous and
Urbanised Latin-American Populations}. Artículo y póster presentados en
el VII Summer Institute of the International Society for Human Ethology
(ISHE), Zadar, Croatia. Póster:
\url{https://jdleongomez.info/en/publication/leongomez2020/PosterA0.pdf}

\textbf{Leongómez, J. D.} (2019, juio). \emph{Diseños Experimentales}.
Taller para la Vicerrectoría de Investigaciones (Universidad El Bosque),
Bogotá, Colombia.

\textbf{Leongómez, J. D.}, Murray, A.K., \& Roberts, S.C. (2018,
septiembre). \emph{Contextualising courtship: Male body odour effects on
vocal modulation}. Artículo presentado en el XXIV Biannual Conference
for Human Ethology (ISHE), Santiago, Chile.

\textbf{Leongómez, J. D.} (2017, septiembre). \emph{Efectos de los
niveles hormonales, masculinidad y feminidad, en la discriminación tonal
de hombres y mujeres.}. Artículo presentado en el XXIII Congreso
Institucional de Investigaciones (Universidad El Bosque), Bogotá,
Colombia.

\textbf{Leongómez, J. D.} (2017, agosto). \emph{The sound of seduction:
Vocal modulations and their effects during courtship}. Sesión plenaria
en el 2do Simposio de Neurociencias, Cognición y Sociedad (Universidad
Javeriana), Bogotá, Colombia.

\textbf{Leongómez, J. D.} (2017, marzo). \emph{Is the relationship
between musical and linguistic processing a vestige of the evolutionary
origins of music?}. Sesión plenaria en el 2017 Brain Awareness Week 2017
(Universidad El Bosque), Bogotá, Colombia.

\textbf{Leongómez, J. D.}, Mileva, V.R., Little, A.C., \& Roberts, S.C.
(2016, agosto). \emph{Perceived differences in social status between
speaker and listener affect the speaker's vocal characteristics}.
Artículo presentado en el XXIII Biennial Congress On Human Ethology
(ISHE), Stirling, Reino Unido.

\textbf{Leongómez, J. D.}, Binter, J., Kubicová, L., Stolařová, P.,
Klapilová, K., Havlíček, J., \& Roberts, S.C. (2014, abril). \emph{Vocal
modulation during courtship increases proceptivity even in naive
listeners}. Artículo presentado en el European Human Behaviour and
Evolution Association (EHBEA) Annual Meeting 2014 (University of
Bristol), Bristol, Reino Unido.

\textbf{Leongómez, J. D.}, \& Roberts, S.C. (2013, abril). \emph{The
sound of sweet nothings: How men and women modulate vocal parameters in
speech directed towards attractive and unattractive individuals}.
Presentación oral para el Mind and Brain Group (University of Stirling),
Stirling, Reino Unido.

\textbf{Leongómez, J. D.}, \& Roberts, S.C. (2013, marzo). \emph{The
sound of sweet nothings: How men and women modulate vocal parameters in
speech directed towards attractive and unattractive individuals}. Póster
presentado en el European Human Behaviour and Evolution Association
(EHBEA) Annual Meeting 2013 (VU University Amsterdam), Amsterdam,
Netherlands.

\textbf{Leongómez, J. D.} (2013, marzo). \emph{Voice research techniques
using Praat}. Taller en la Facultad de Humanidades (Charles University),
Praga, República Checa.

\textbf{Leongómez, J. D.} (2013, marzo). \emph{Acoustic Communication}.
Presentación oral en el Centre for Theoretical Study, Praga, República
Checa.

\textbf{Leongómez, J. D.}, \& Roberts, S.C. (2012, mayo). \emph{The
sound of sweet nothings: How men and women modulate vocal parameters in
speech directed towards attractive and unattractive individuals}.
Presentación oral para el Group of Human Ethology (Charles University),
Praga, República Checa.

\textbf{Leongómez, J. D.} (2013, marzo). \emph{Voice research techniques
using Praat}. Taller en Scottish Universities Psychology Postgraduate
Research Training -- SUPPORT (University of Stirling), Stirling, Reino
Unido.

\textbf{Leongómez, J. D.}, \& Roberts, S.C. (2011, febrero). \emph{The
sound of sweet nothings: How men and women modulate vocal parameters in
speech directed towards attractive and unattractive individuals}.
Presentación para el Behaviour and Evolution Research Group (University
of Stirling), Stirling, Reino Unido.

\textbf{Leongómez, J. D.}, \& Roberts, S.C. (2010, diciembre).
\emph{Context-dependant modulation of vocal parameters}. Póster
presentado en el Language of Music, the Music of Language; Meeting 2:
Entrainment/Meaning (Centre for Music \& Science, University of
Cambridge), Cambridge, Reino Unido.

\endgroup

\hypertarget{asociaciones}{%
\section{Asociaciones}\label{asociaciones}}

\begin{cventries}
    \cventry{\href{https://ishe.org/}{International Society for Human Ethology} (ISHE)}{}{}{2016 - Actualmente}{}\vspace{-4.0mm}
    \cventry{\href{https://www.psychologicalscience.org/}{Association for Psychological Science} (APS)}{}{}{2013 - 2020}{}\vspace{-4.0mm}
    \cventry{\href{https://www.cambridge.org/core/membership/ehbea}{European Human Behaviour and Evolution Association} (EHBEA)}{}{}{2013 - 2016}{}\vspace{-4.0mm}
\end{cventries}

\hypertarget{roles-editoriales}{%
\section{Roles Editoriales}\label{roles-editoriales}}

\begin{footnotesize}
\textbf{Nota:} Mi registro verificado como par académico y editor internacional está disponible en mi perfil de \href{https://www.webofscience.com/wos/author/record/387716}{Web of Science}.
\end{footnotesize}

\begin{cventries}
    \cventry{Recommender (editor)}{\href{https://rr.peercommunityin.org/}{PCI Registered Reports}}{Recommender}{Desde 2023}{\begin{cvitems}
\item Asignación de pares, evaluación de propuestas y emisión de recomendaciones de las fases 1 y 2 de \href{https://www.cos.io/initiatives/registered-reports}{reportes registrados}
\item Perfil como \href{https://rr.peercommunityin.org/public/user_public_page?userId=1996}{recomendador}
\end{cvitems}}
    \cventry{Guest Editor}{\href{https://royalsocietypublishing.org/journal/rstb}{Philosophical Transactions of the Royal Society B: Biological Sciences}}{Comité Editorial invitado}{2020 - 2021}{\begin{cvitems}
\item Theme Issue \textbf{\textit{Voice modulation: from origin and mechanism to social impact}} (\href{https://royalsocietypublishing.org/toc/rstb/2021/376/1840}{\textbf{Parte 1}}, \href{https://royalsocietypublishing.org/toc/rstb/2022/377/1841}{\textbf{Parte 2}})
\item Editado por \textbf{Juan David Leongómez}, Katarzyna Pisanski, David Reby, Disa Sauter, Nadine Lavan, Marcus Perlman y Jaroslava Varella Valentova
\end{cvitems}}
    \cventry{Review Editor}{\href{https://www.frontiersin.org/journals/psychology}{Frontiers in Psychology}}{Equipo editorial}{2019 - Actualmente}{\begin{cvitems}
\item Specialty section on \href{https://www.frontiersin.org/journals/psychology/sections/evolutionary-psychology}{Evolutionary Psychology}
\item Perfil \href{https://loop.frontiersin.org/people/438954/overview}{Loop}
\end{cvitems}}
    \cventry{Journals Incluyen}{Par \textit{Ad Hoc}}{\space}{\space}{}\vspace{-4.0mm}
\end{cventries}
\vspace{-3mm}
\begin{footnotesize}
  \begin{minipage}[c]{0.32\linewidth}
    • \href{https://www.frontiersin.org/journals/psychology}{Frontiers in Psychology} (Q1)\newline
    • \href{https://www.sciencedirect.com/journal/cortex}{Cortex} (Q1)\newline
    • \href{https://www.springer.com/journal/10919/}{Journal of Nonverbal Behavior} (Q1)\newline
    • \href{https://journals.sagepub.com/home/pec}{Perception} (Q1)\newline
    • \href{https://www.tandfonline.com/journals/hbas20}{Basic and Applied Social Psychology} (Q1)\newline
  \end{minipage}\begin{minipage}[c]{0.45\linewidth}
    • \href{https://royalsocietypublishing.org/journal/rspb}{Proceedings of the Royal Society B: Biological Sciences} (Q1)\newline
    • \href{https://royalsocietypublishing.org/journal/rsos}{Royal Society Open Science} (Q1)\newline
    • \href{https://journals.plos.org/plosone/}{PLOS ONE} (Q1)\newline
    • \href{https://royalsocietypublishing.org/journal/rsbl}{Biology Letters} (Q1)\newline
    • \href{https://www.sciencedirect.com/journal/evolution-and-human-behavior}{Evolution and Human Behavior} (Q1)\newline
  \end{minipage}\begin{minipage}[c]{0.23\linewidth}
    • \href{https://www.nature.com/srep/}{Scientific Reports} (Q1)\newline
    • \href{https://journals.sagepub.com/home/evp}{Evolutionary Psychology} (Q2)\newline
    • \href{http://sumapsicologica.konradlorenz.edu.co/}{Suma Psicológica} (Q3)\newline
    • \href{https://ishe.org/human-ethology/}{Human Ethology}\newline
    • \href{https://ishe.org/human-ethology-bulletin-2006-2011/}{Human Ethology Bulletin}
  \end{minipage}
\end{footnotesize}

\hypertarget{publicaciones}{%
\section{Publicaciones}\label{publicaciones}}

\hypertarget{artuxedculos-en-revistas-cientuxedficas}{%
\subsection{Artículos en Revistas
Científicas}\label{artuxedculos-en-revistas-cientuxedficas}}

\begin{tcolorbox}[enhanced,
        on line, 
        boxsep=4pt, left=0pt,right=0pt,top=0pt,bottom=0pt,
        colframe=white,colback=blue,
        hyperurl={https://scholar.google.com/citations?user=8Q0jKHsAAAAJ}]
  
\color{white}
\begin{minipage}[c]{0.15\linewidth}
  \begin{center} \begin{huge} 12 \end{huge}
  \begin{small} Índice \textit{h} \end{small} \end{center} 
\end{minipage} \begin{minipage}[c]{0.15\linewidth}
  \begin{center} \begin{huge} 24 \end{huge}
  \begin{small} Índice \textit{g} \end{small} \end{center}
\end{minipage} \begin{minipage}[c]{0.17\linewidth}
  \begin{center} \begin{huge} 17 \end{huge}
  \begin{small} Índice i10 \end{small} \end{center}
\end{minipage} \begin{minipage}[c]{0.20\linewidth}
  \begin{center} \begin{huge} 30 \end{huge}
  \begin{small} Artículos \end{small} \end{center}
\end{minipage} \begin{minipage}[c]{0.12\linewidth}
  \begin{center} \begin{huge} 20 \end{huge}
  \begin{small} en Q1 \end{small} \end{center}
\end{minipage} \begin{minipage}[c]{0.18\linewidth}  
  \begin{center} \begin{huge} 623 \end{huge} 
  \begin{small} Citas \end{small} \end{center}
\end{minipage}
\begin{center}
\noindent\rule{8cm}{0.4pt}
\end{center}
\begin{center}
\footnotesize{En la mayoría de los artículos he realizado los análisis estadísticos}\end{center}
\end{tcolorbox}

\begingroup
\footnotesize
\setlength{\parindent}{-0.5in}
\setlength{\leftskip}{0.5in}

Martínez, C. A., \& \textbf{Leongómez, J. D.} (in press). Dehumanizing
framing effects on agency, punishment, and re-socialization attributions
towards ex-perpetrators in post-conflict. \emph{Peace and Conflict:
Journal of Peace Psychology}. \textbf{(Q2; índice \emph{h}: 39)}

Kleisner, K., Tureček, P., Saribay, S. A., Pavlovič, O.,
\textbf{Leongómez, J. D.}, Roberts, S. C., Havlíček, J., Valentova, J.
V., Apostol, S., Akoko, R. M., \& Varella, M. A. C. (2023).
Distinctiveness and femininity, rather than symmetry and masculinity,
affect facial attractiveness across the world . \emph{Evolution and
Human Behavior}. Advance online publication.
\href{https://authors.elsevier.com/c/1hyfr3tz492kal}{https://doi.org/10.1016/j.evolhumbehav.2023.10.001}
\textbf{(Q1; índice \emph{h}: 123)}

Vásquez-Amézquita, M., \textbf{Leongómez, J. D.}, Salvador, A., \& Seto,
M. C. (2023). What can the eyes tell us about atypical sexual
preferences as a function of sex and age? Linking eye movements with
child-related chronophilias. \emph{Forensic Sciences Research, 8}, 5-15.
\url{https://doi.org/10.1093/fsr/owad009} \textbf{(Q1; índice \emph{h}:
22)}

Terry, J., Ross, R. M., Nagy, T., \ldots, \textbf{Leongómez, J. D.},
\ldots, \& Field, A. P. (2023). Data from an international multi-centre
study of statistics and mathematics anxieties and related variables in
university students (the SMARVUS dataset). \emph{Journal of Open
Psychology Data, 11}(1). \url{https://doi.org/10.5334/jopd.80}
\textbf{(NA; \emph{Journal} nuevo)}

\textbf{Leongómez, J. D.}, Havlíček, J., \& Roberts, S. C. (2022).
Musicality in human vocal communication: An evolutionary perspective.
\emph{Philosophical Transactions of the Royal Society B: Biological
Sciences, 377}, 20200391. \url{https://doi.org/10.1098/rstb.2020.0391}
\textbf{(Q1; índice \emph{h}: 305)}

Santamaría-García, H., Burgaleta, M., Legaz, A., Flichtentrei, D.,
Cordoba-Delgado, M., Molina, J., Linares, J., Montealegre, J.,
Castelblanco Toro, S., Schulte, M., Paramo, J., Mondragón, I.,
\textbf{Leongómez, J. D.}, Salamone, P., González‑Pacheco, J., Baez, S.,
\& Ibañez, A.(2022). The price of prosociality in pandemic times.
\emph{Humanities and Social Sciences Communications, 9}, 15.
\url{https://doi.org/10.1057/s41599-021-01022-2} \textbf{(Q1; índice
\emph{h}: 21)}

Vásquez-Amézquita, M., Salvador, A., \& \textbf{Leongómez, J. D.}
(2022). Is digit ratio (2D:4D) different between sexual and non-sexual
offenders, and non-offending men? Study of a Colombian sample {[}Existen
diferencias en la ratio 2D:4D entre delincuentes sexuales y no sexuales,
y hombres no delincuentes? Un estudio en una muestra colombiana{]}.
\emph{Interdisciplinaria. Revista de Psicología y Ciencias Afines,
39}(1), 127--141. \url{https://doi.org/10.16888/interd.2022.39.1.8}
\textbf{(Q3; índice \emph{h}: 13)}

Watkins, C., Bovet, J., Fernandez, A. M., \textbf{Leongómez, J. D.},
Zelazniewicz, A., Correa Varella, M. A., \& Wagstaff, D. (2022). Men say
`I love you' before women do: Robust across several countries.
\emph{Journal of Social and Personal Relationships, 39}(7), 2134--2153.
\url{https://doi.org/10.1177/02654075221075264} \textbf{(Q1; índice
\emph{h}: 96)}

Fiala, V., Třebický, V., Pazhoohi, F., \textbf{Leongómez, J. D.},
Tureček, P., Saribay, S. A., \ldots{} Kleisner, K. (2021). Facial
attractiveness and preference of sexual dimorphism: A comparison across
five populations. \emph{Evolutionary Human Sciences, 3}, e38.
\url{https://doi.org/10.1017/ehs.2021.33} \textbf{(Q1; índice \emph{h}:
10)}

Jones, B. C., DeBruine, L. M., Flake, J. K., \ldots,
\textbf{Leongómez, J. D.}, \ldots, \& Chartier, C. R. (2021). To which
world regions does the valence-dominance model of social perception
apply? \emph{Nature Human Behaviour, 5}, 159--169.
\url{https://doi.org/10.1038/s41562-020-01007-2} \textbf{(Q1; índice
\emph{h}: 79)}

Kleisner, K., \textbf{Leongómez, J. D.}, Pisanski, K., Fiala, V.,
Cornec, C., Groyecka, A., \ldots{} Akoko, R. M. (2021). Predicting
strength from aggressive vocalisations versus speech in African bushland
and urban communities. \emph{Philosophical Transactions of the Royal
Society B: Biological Sciences, 376}, 20200403.
\url{https://doi.org/10.1098/rstb.2020.0403} \textbf{(Q1; índice
\emph{h}: 305)}

Kleisner, K., Tureček, P., Roberts, S. C., Havlíček, J., Valentova, J.
V., Akoko, R. M., \textbf{Leongómez, J. D.}, Apostol, S., Varella, M. A.
C., \& Saribay, S. A. (2021). How and why patterns of sexual dimorphism
in human faces vary across the world. \emph{Scientific Reports, 11},
5978. \url{https://doi.org/10.1038/s41598-021-85402-3} \textbf{(Q1;
índice \emph{h}: 282)}

\textbf{Leongómez, J. D.}, Pisanski, K., Reby, D., Sauter, D., Lavan,
N., Perlman, M., \& Varella Valentova, J. (2021). Voice modulation: From
origin and mechanism to social impact. \emph{Philosophical Transactions
of the Royal Society B: Biological Sciences, 376}, 20200386.
\url{https://doi.org/10.1098/rstb.2020.0386} \textbf{(Q1; índice
\emph{h}: 305)}

\textbf{Leongómez, J. D.}, Sánchez, O. R., Vásquez-Amézquita, M., \&
Roberts, S. C. (2021). Contextualising courtship: Exploring male body
odour effects on vocal modulation. \emph{Behavioural Processes, 193},
104531. \url{https://doi.org/10.1016/j.beproc.2021.104531} \textbf{(Q2;
índice \emph{h}: 87)}

\textbf{Leongómez, J. D.}, Sánchez, O. R., Vásquez-Amézquita, M.,
Valderrama, E., Castellanos-Chacón, A., Morales-Sánchez, L., \ldots{}
González-Santoyo, I. (2020). Self-reported health is related to body
height and waist circumference in rural indigenous and urbanised
Latin-American populations. \emph{Scientific Reports, 10}, 4391.
\url{https://doi.org/10.1038/s41598-020-61289-4} \textbf{(Q1; índice
\emph{h}: 282)}

Bonilla, F. M., \& \textbf{Leongómez, J. D.} (2019). Efectos de la
expresión emocional y de la orientación del rostro sobre las respuestas
conductuales y el componente N170. \emph{Psychologia, 13}(2), 95--106.
\url{https://doi.org/10.21500/19002386.3473}\textbf{(Q4; índice
\emph{h}: 28)}

Vásquez Amézquita, M., \textbf{Leongómez, J. D.}, Seto, M. C., \&
Salvador, A. (2019). Differences in visual attention patterns to
sexually mature and immature stimuli between heterosexual sexual
offenders, nonsexual pffenders, and nonoffending men. \emph{Journal of
Sex Research, 56}(2), 213--228.
\url{https://doi.org/10.1080/00224499.2018.1511965} \textbf{(Q1; índice
\emph{h}: 124)}

Vásquez-Amézquita, M., \textbf{Leongómez, J. D.}, Seto, M. C., Bonilla,
F. M., Rodríguez-Padilla, A., \& Salvador, A. (2019). Visual attention
patterns differ in gynephilic and androphilic men and women depending on
age and gender of targets. \emph{Journal of Sex Research, 56}(1),
85--101. \url{https://doi.org/10.1080/00224499.2017.1372353}
\textbf{(Q1; índice \emph{h}: 124)}

Watkins, C. D., \textbf{Leongómez, J. D.}, Bovet, J., Żelaźniewicz, A.,
Korbmacher, M., Corrêa Varella, M. A., \ldots{} Bolgan, S. (2019).
National income inequality predicts cultural variation in mouth to mouth
kissing. \emph{Scientific Reports, 9}, 6698.
\url{https://doi.org/10.1038/s41598-019-43267-7} \textbf{(Q1; índice
\emph{h}: 282)}

Culpepper, P. D., Havlíček, J., \textbf{Leongómez, J. D.}, \& Roberts,
S. C. (2018). Visually activating pathogen disgust: A new instrument for
studying the behavioral immune system. \emph{Frontiers in Psychology,
9}, 1397. \url{https://doi.org/10.3389/fpsyg.2018.01397} \textbf{(Q1;
índice \emph{h}: 157)}

Danel, D. P., Valentova, J. V., Sánchez, O. R.,
\textbf{Leongómez, J. D.}, Corrêa Varella, M. A., \& Kleisner, K.
(2018). A cross-cultural study of sex-typicality and averageness:
Correlation between frontal and lateral measures of human faces.
\emph{American Journal of Human Biology, 30}(5), e23147.
\url{https://doi.org/10.1002/ajhb.23147} \textbf{(Q1; índice \emph{h}:
89)}

Vásquez-Amézquita, M., \textbf{Leongómez, J. D.}, Seto, M. C., Bonilla,
F. M., Rodríguez-Padilla, A., \& Salvador, A. (2018). No relation
between digit ratio (2D:4D) and visual attention patterns to sexually
preferred and non-preferred stimuli. \emph{Personality and Individual
Differences, 120}, 151--158.
\url{https://doi.org/10.1016/j.paid.2017.08.022} \textbf{(Q1; índice
\emph{h}: 193)}

Bonilla, F. M., \& \textbf{Leongómez, J. D.} (2017). Efectos en la
amplitud y la latencia del componente N170 ante la presentación de
rostros emocionales de ira y miedo. \emph{Psychologia, 11}(1), 39--48.
\url{https://doi.org/10.21500/19002386.3100} \textbf{(Q4; índice
\emph{h}: 28)}

\textbf{Leongómez, J. D.}, Mileva, V. R., Little, A. C., \& Roberts, S.
C. (2017). Perceived differences in social status between speaker and
listener affect the speaker's vocal characteristics. \emph{PLOS One,
12}(6), e0179407. \url{https://doi.org/10.1371/journal.pone.0179407}
\textbf{(Q1; índice \emph{h}: 404)}

Cobey, K. D., Nicholls, M., \textbf{Leongómez, J. D.}, \& Roberts, S. C.
(2015). Self-reported dominance in women: Associations with hormonal
contraceptive use, relationship status, and testosterone. \emph{Adaptive
Human Behavior and Physiology, 1}(4), 449--459.
\url{https://doi.org/10.1007/s40750-015-0022-8} \textbf{(Q3; índice
\emph{h}: 22)}

\textbf{Leongómez, J. D.} (2015). La música como objeto de estudio
científico: consideraciones en torno a la musicalidad y el origen de la
música. \emph{(Pensamiento), (Palabra) y Obra, 13}, 77--86.
\url{https://doi.org/10.17227/2011804X.15PPO77.86} \textbf{(NA)}

\textbf{Leongómez, J. D.}, Binter, J., Kubicová, L., Stolařová, P.,
Klapilová, K., Havlíček, J., \& Roberts, S. C. (2014). Vocal modulation
during courtship increases proceptivity even in naive listeners.
\emph{Evolution and Human Behavior, 35}(6), 489--496.
\url{https://doi.org/10.1016/j.evolhumbehav.2014.06.008} \textbf{(Q1;
índice \emph{h}: 123)}

Binter, J., \textbf{Leongómez, J. D.}, Moyano, N., Valentova, J., Jouza,
L., \& Klapilová, K. (2012). Sex differences in the incidence of sexual
fantasies focused on evolutionary relevant objects. \emph{Anthropologie,
50}(1), 83--93. \textbf{(Q3; índice \emph{h}: 9)}

Ferdenzi, C., Lemaître, J.-F., \textbf{Leongómez, J. D.}, \& Roberts, S.
C. (2011). Digit ratio (2D:4D) predicts facial, but not voice or body
odour, attractiveness in men. \emph{Proceedings of the Royal Society B:
Biological Sciences, 278}(1724), 3551--3557.
\url{https://doi.org/10.1098/rspb.2011.0544} \textbf{(Q1; índice
\emph{h}: 279)}

\textbf{Leongómez, J. D.} (2008). El origen no humano de la música.
\emph{(Pensamiento), (Palabra) y Obra, 1}, 87--97.
\url{https://revistas.pedagogica.edu.co/index.php/revistafba/article/view/50}
\textbf{(NA)}

\endgroup

\blacktriangleright\blacktriangleright\blacktriangleright\space \textbf{\textit{Preprints} y Artículos en Revisión}

\begingroup
\footnotesize
\setlength{\parindent}{-0.5in}
\setlength{\leftskip}{0.5in}

Hadavi, S., Kuroda, J., Shimozono, T., \textbf{Leongómez, J. D.}, \&
Savage, P. E. (2023). Cross-cultural relationships between music,
emotion, and visual imagery: A comparative study of Iran, Canada, and
Japan {[}Stage 1 Registered Report{]}. \emph{PsyArXiv}.
\url{https://doi.org/10.31234/osf.io/26yg5}

\endgroup

\hypertarget{guuxedas-metodoluxf3gicas}{%
\subsection{Guías metodológicas}\label{guuxedas-metodoluxf3gicas}}

\begingroup
\footnotesize
\setlength{\parindent}{-0.5in}
\setlength{\leftskip}{0.5in}

\textbf{Leongómez, J. D.} (2023). Meta-análisis de correlaciones y
meta-regresión en R: Guía práctica. \emph{MetaArXiv}.
\url{https://doi.org/10.31222/osf.io/yaxd4}

\textbf{Leongómez, J. D.} (2020). Análisis de poder estadístico y
cálculo de tamaño de muestra en R: Guía práctica. \emph{Zenodo}.
\url{https://doi.org/10.5281/zenodo.3988776}

\endgroup

\hypertarget{articulos-de-divulgaciuxf3n}{%
\subsection{Articulos de
Divulgación}\label{articulos-de-divulgaciuxf3n}}

\begingroup
\footnotesize
\setlength{\parindent}{-0.5in}
\setlength{\leftskip}{0.5in}

Mileva, V. R., \& \textbf{Leongómez, J. D.} (2018). Want to know if your
partner's cheating on you? Just listen to their voice. \emph{The
Conversation}.
\url{https://theconversation.com/want-to-know-if-your-partners-cheating-on-you-just-listen-to-their-voice-92387}

Mileva, V. R., \& \textbf{Leongómez, J. D.} (2017). We change our voice
when we talk to high-status people, shows new study. \emph{The
Conversation}.
\url{https://theconversation.com/we-change-our-voice-when-we-talk-to-high-status-people-shows-new-study-80053}

\endgroup

\hypertarget{publicaciones-en-otros-formatos}{%
\subsection{Publicaciones en otros
formatos}\label{publicaciones-en-otros-formatos}}

\begin{minipage}[c]{0.10\linewidth}
\href{https://www.youtube.com/@InvestigacionAbierta}{\includegraphics[width=1.7cm, height=1.7cm]{Logo_IA.png}}
\end{minipage} \begin{minipage}[c]{0.90\linewidth} \begin{footnotesize}
\textcolor{red}{\faYoutube} \href{https://www.youtube.com/@InvestigacionAbierta}{Investigación Abierta} es un canal de YouTube donde publico videos y tutoriales acerca de métodos y buenas prácticas de investigación, estadística y ciencia abierta, así como información sobre algunos programas útiles de código abierto.
\end{footnotesize}
\end{minipage}

\begingroup
\footnotesize
\setlength{\parindent}{-0.5in}
\setlength{\leftskip}{0.5in}

\textbf{Leongómez, J. D.} {[}Investigación Abierta{]}. (2023).
\emph{Meta-análisis de correlaciones en R} {[}Video{]}. YouTube.
\url{https://www.youtube.com/watch?v=RNw92QdsW4g}

\textbf{Leongómez, J. D.} {[}Investigación Abierta{]}. (2022).
\emph{Ciencia abierta: necesidad, principios e iniciativas} {[}Video{]}.
YouTube. hhttps://www.youtube.com/watch?v=Fl98MoBqSaI

\textbf{Leongómez, J. D.} {[}Investigación Abierta{]}. (2021).
\emph{Hacer meta-análisis en jamovi es muy fácil} {[}Video{]}. YouTube.
\url{https://www.youtube.com/watch?v=ntBbkOn9D_o}

\textbf{Leongómez, J. D.} {[}Investigación Abierta{]}. (2020).
\emph{Poder estadístico y tamaño de muestra en R} {[}3 Videos{]}.
YouTube.
\url{https://www.youtube.com/playlist?list=PLHk7UNt35ccVdyHqnQ6oXVYA6JBNFrE1x}

\textbf{Leongómez, J. D.} {[}Investigación Abierta{]}. (2020).
\emph{¿Qué es un valor p? Te lo puedo mostrar gráficamente y con
simulaciones} {[}Video{]}. YouTube.
\url{https://www.youtube.com/watch?v=X0At_roDnHc}

\textbf{Leongómez, J. D.} {[}Investigación Abierta{]}. (2020).
\emph{Introducción a jamovi} {[}2 Videos{]}. YouTube.
\url{https://www.youtube.com/playlist?list=PLHk7UNt35ccXX4I61PiVMOf9VkijgaJN8}

\textbf{Leongómez, J. D.} {[}Investigación Abierta{]}. (2020).
\emph{Cómo grabar clases y tutoriales usando OBS Studio} {[}Video{]}.
YouTube. \url{https://www.youtube.com/watch?v=05f77YdTb1E}

\endgroup

\begin{minipage}[c]{0.10\linewidth}
\href{https://jdleongomez.info/es/#shiny}{\includegraphics[width=1.5cm, height=1.74cm]{shiny_hex.png}}
\end{minipage} \begin{minipage}[c]{0.90\linewidth} \begin{footnotesize}
He creado algunas aplicaciones con fines educativos usando \href{https://shiny.posit.co/}{Shiny} en R. Aunque hay varias en desarrollo (aún privadas), en esta sección hay algunos ejemplos de aplicaciones públicas.
\end{footnotesize}
\end{minipage}

\begingroup
\footnotesize
\setlength{\parindent}{-0.5in}
\setlength{\leftskip}{0.5in}

\textbf{Leongómez, J. D.} (2023). \emph{PowerSimulate} {[}Colección de
aplicaciones{]}. \url{https://shiny.jdl-svr.lat/PowerSimulate/}

\textbf{Leongómez, J. D.} (2023). \emph{PowerSimulate: Correlación}
{[}Software{]}. \url{https://shiny.jdl-svr.lat/PowerSimulate_corr_ES/}
(Versión en inglés:
\url{https://shiny.jdl-svr.lat/PowerSimulate_corr_EN/})

\textbf{Leongómez, J. D.} (2023). \emph{PowerSimulate: Prueba t
independiente} {[}Software{]}.
\url{https://shiny.jdl-svr.lat/PowerSimulate_ind_t_ES/} (Versión en
inglés: \url{https://shiny.jdl-svr.lat/PowerSimulate_ind_t_EN/})

\textbf{Leongómez, J. D.} (2023). \emph{PowerSimulate: Prueba t pareada}
{[}Software{]}. \url{https://shiny.jdl-svr.lat/PowerSimulate_pair_t_ES/}
(Versión en inglés:
\url{https://shiny.jdl-svr.lat/PowerSimulate_pair_t_EN/})

\textbf{Leongómez, J. D.} (2023). \emph{\textbf{ScolarlyOutput}: Crea y
exporta un gráfico completo de tu perfil de Google académico} (Versión
1.0.0) {[}Software{]}.
\href{https://shiny.jdl-svr.lat/ScholarlyOutput/}{https://doi.org/10.5281/zenodo.8060475}
(Versión en español:
\url{https://shiny.jdl-svr.lat/ScholarlyOutput_ES/}; Versión en inglés:
\url{https://shiny.jdl-svr.lat/ScholarlyOutput/})

\textbf{Leongómez, J. D.} (2021). \emph{\textbf{ProbDnD}: Probabilidad
de obtener un valor en DnD, combinando múltiples dados y un modificador}
{[}Software{]}. \url{https://shiny.jdl-svr.lat/ProbDnD/} (Versión en
inglés: \url{https://shiny.jdl-svr.lat/ProbDnD_EN/})

\endgroup

\hypertarget{supervisiuxf3n-de-investigaciuxf3n}{%
\section{Supervisión de
Investigación}\label{supervisiuxf3n-de-investigaciuxf3n}}

\hypertarget{section}{%
\subsection{\texorpdfstring{\textbf{Posgrado}}{}}\label{section}}

\begin{footnotesize}
La Facultad de Psicología de la Universidad El Bosque no ofrece títulos de investigación, y todos los cursos de nivel de maestrías son profesionales. Debido a esto, las oportunidades de supervisar a estudiantes de posgrado son limitadas, y la mayor parte de mi supervisión de posgrado ha sido externa.
\end{footnotesize}

\begin{cventries}
    \cventry{PhD in Neuroscience}{\href{https://www.researchgate.net/profile/Milena-Vasquez-Amezquita}{Milena Vásquez-Amézquita}}{\href{https://www.uv.es/}{Universitat de València}, España}{2015 - 2018}{\begin{cvitems}
\item Tésis \textbf{\textit{(Summa Cum Laude)}}: \textit{\href{http://hdl.handle.net/10550/67639}{Preferencias sexuales típicas y atípicas según sexo y edad de los estímulos: Utilidad de la técnica de rastreo ocular} [Typical and atypical sexual preferences according to sex and age of the stimuli: Usefulness of the eye tracking technique]}
\item Supervisión conjunta con  Alicia Salvador
\end{cvitems}}
    \cventry{Professional Doctorate in Counselling Psychology}{\href{https://www.researchgate.net/profile/Francisco-Flores-14}{Francisco Javier Flores}}{\href{https://www.uel.ac.uk/}{U. of East London}, Reino Unido}{2015 - 2018}{\begin{cvitems}
\item Tésis: \textit{ What sense do people make of the functions of their ’behaviours that may be causing problems in their everyday life’? A hybrid deductive/inductive template analysis}
\item Supervisión conjunta con Lisa Chiara Fellin
\end{cvitems}}
    \cventry{Maestría en Psicología}{Adrián Acosta Guerrero}{\href{https://www.unbosque.edu.co/}{Universidad El Bosque}, Colombia}{2019 - 2020}{\begin{cvitems}
\item Trabajo de grado \textbf{\textit{(Meritorio)}}: \textit{\href{http://hdl.handle.net/20.500.12495/4416}{La voz como predictor de sintomatología asociada a depresión y ansiedad} [Voice as a predictor of symptomatology associated with depression and anxiety]}
\item Supervisión conjunta con \href{https://www.researchgate.net/profile/Milena-Vasquez-Amezquita}{Milena Vásquez-Amézquita}
\end{cvitems}}
    \cventry{Psychological Research Methods (Evolutionary Psychology) MSc}{Julia Sanz-Vidania}{\href{https://www.stir.ac.uk/}{University of Stirling}, Reino Unido}{2013 - 2014}{\begin{cvitems}
\item Trabajo de grado \textbf{\textit{(Meritorio)}}: \textit{Sexy Chat: Mate-Choice Preferences for Speech Content in the Absence of Auditory Cues}
\item Supervisión conjunta con \href{https://www.scraigroberts.com/}{S Craig Roberts}
\end{cvitems}}
\end{cventries}

\hypertarget{section-1}{%
\subsection{\texorpdfstring{\textbf{Pregrado}}{}}\label{section-1}}

\begin{footnotesize}
Los estudiantes universitarios supervisados que figuran a continuación proceden de distintas universidades y programas académicos. Esta lista sólo incluye a los estudiantes de quienes fui supervisor principal u obtuvieron mención a trabajo de grado meritorio. Adicionalmente, actualmente estoy supervisando seis estudiantes como supervisor principal, y he supervisado a más de 25 estudiantes como co-supervisor.

\textbf{Nota:} Los hipervínculos en los títulos de algunos trabajos de grado recientes conducen a videos cortos (< 10 min) de socialización de los resultados, que pueden ser consultados para constatar su nivel, incluyendo los análisis estadísticos.
\end{footnotesize}

\begin{cventries}
    \cventry{Psicología}{\href{https://www.linkedin.com/in/andres-castellanos-chacon/}{Andrés Castellanos-Chacón}}{\href{https://www.unbosque.edu.co/}{Universidad El Bosque}, Colombia}{Desde 2017}{\begin{cvitems}
\item Supevisión práctica profesional en investigación (2017 - 2018)
\item Proyecto: \textit{Señales perceptibles de salud física y mental en rostros, voces y olores corporales, y su relación con niveles hormonales}
\item Supervisión docente (Desde 2019)
\end{cvitems}}
    \cventry{Psicología}{Isabella Russo}{\href{https://www.unbosque.edu.co/}{Universidad El Bosque}, Colombia}{2023}{\begin{cvitems}
\item Trabajo de grado: \textit{\href{https://youtu.be/nX3p5Bt_vjQ}{¿El género, preferenca sexual y nivel de sociosexualidad de una persona podrían influir en su capacidad para detectar la sociosexualidad de otros a través de rostros?}}
\end{cvitems}}
    \cventry{Psicología}{Maria Camila Wilches y Johan Sebatián Castiblanco}{\href{https://www.unbosque.edu.co/}{Universidad El Bosque}, Colombia}{2022 - 2023}{\begin{cvitems}
\item Trabajo de grado: \textit{\href{https://youtu.be/FlZvukFqTcc}{El rol del género en la identificación de la sociosexualidad a partir de las voces}}
\end{cvitems}}
    \cventry{Psicología}{Angie Alejandra Lozano Sanjuan, Daniela Martínez Franco y Mariana Saavedra Botero}{\href{https://www.unbosque.edu.co/}{Universidad El Bosque}, Colombia}{2021 - 2022}{\begin{cvitems}
\item \textbf{\textit{Trabajo de grado meritorio}}: \textit{\href{https://youtu.be/A9xNV3BqRJw}{¿Somos capaces de detectar qué tan sociosexual es una persona a partir de su voz y/o su rostro?}}
\end{cvitems}}
    \cventry{Psicología}{John Jairo Rubio}{\href{https://www.unbosque.edu.co/}{Universidad El Bosque}, Colombia}{2021 - 2022}{\begin{cvitems}
\item Trabajo de grado: \textit{\href{https://youtu.be/G9eqxpyKF5A}{¿Existe relación entre la simetría facial y el atractivo de la voz?} }
\end{cvitems}}
    \cventry{Psicología}{Maria Daniela Martínez Luna y Juan Sebastián Preciado Ruíz}{\href{https://www.unbosque.edu.co/}{Universidad El Bosque}, Colombia}{2020 - 2021}{\begin{cvitems}
\item \textbf{\textit{Trabajo de grado meritorio}}: \textit{\href{https://youtu.be/BNNseX-PK7s}{¿Existe relación entre la sociosexualidad y la voz?}}
\end{cvitems}}
    \cventry{Psicología}{Maria Paula Moreno Rodríguez y Andrés Felipe Orozco Serrato}{\href{https://www.unbosque.edu.co/}{Universidad El Bosque}, Colombia}{2019 - 2021}{\begin{cvitems}
\item Trabajo de grado: \textit{\href{https://youtu.be/qHn87CoBEm4}{¿Existe relación entre la forma del rostro y la socio-sexualidad?}}
\end{cvitems}}
    \cventry{Psicología}{Danny Ferley Gaitan Rodríguez y Hasbleidy Gamboa Ordoñez}{\href{https://www.unbosque.edu.co/}{Universidad El Bosque}, Colombia}{2019 - 2020}{\begin{cvitems}
\item Trabajo de grado: \textit{\href{https://youtu.be/J6nUuifYjbU}{Detección de infidelidad y sociosexualidad a partir de rostros: Análisis preliminar}}
\end{cvitems}}
    \cventry{Pedagogía Musical}{Natalia Elízabeth Moreno Buitrago y Juan Felipe Pérez Ariza}{\href{https://www.upn.edu.co/}{U. Pedagógica Nacional}, Colombia}{2017 - 2019}{\begin{cvitems}
\item \textbf{\textit{Trabajo de grado meritorio}}: \textit{\href{http://hdl.handle.net/20.500.12209/10443}{Musicalidad y cohesión social: una aproximación experimental y bibliográfica desde el trabajo en equipo}}
\end{cvitems}}
    \cventry{Psicología}{Paula Andrea Betancourt Velandia y Ana Sofía Gómez Castelblanco}{\href{https://www.unbosque.edu.co/}{Universidad El Bosque}, Colombia}{2018 - 2019}{\begin{cvitems}
\item Trabajo de grado: \textit{Hombres que alcanzan voces menos agudas y con un volumen más bajo, tienden a tener más relaciones sexuales casuales}
\end{cvitems}}
    \cventry{Biología}{Maria Alejandra Abello Mozo}{\href{https://www.unbosque.edu.co/}{Universidad El Bosque}, Colombia}{2017 - 2018}{\begin{cvitems}
\item \textbf{\textit{Trabajo de grado meritorio}}: \textit{Desarrollo y evaluación de una metodología nueva para manipular las variables del atractivo, dominancia y sexo simultáneamente en fotos de caras humanas con el programa PsychoMorph}
\end{cvitems}}
    \cventry{Psicología}{Cindy Paola Moncada Gómez}{\href{https://www.unbosque.edu.co/}{Universidad El Bosque}, Colombia}{2017 - 2018}{\begin{cvitems}
\item Trabajo de grado: \textit{La voz del sexo casual: ¿las características vocales predicen la disposición al sexo sin compromiso en hombres y mujeres? A}
\end{cvitems}}
    \cventry{Psicología}{Laura Milena Estupiñan Aldana}{\href{https://www.unbosque.edu.co/}{Universidad El Bosque}, Colombia}{2017 - 2018}{\begin{cvitems}
\item Trabajo de grado: \textit{La voz del sexo casual: ¿las características vocales predicen la disposición al sexo sin compromiso en hombres y mujeres? B}
\end{cvitems}}
    \cventry{Psicología}{Vanesa Díaz Güiza}{\href{https://www.unbosque.edu.co/}{Universidad El Bosque}, Colombia}{2016 - 2018}{\begin{cvitems}
\item Trabajo de grado: \textit{La voz del sexo casual: ¿las características vocales predicen la disposición al sexo sin compromiso en hombres y mujeres? C}
\end{cvitems}}
    \cventry{Psicología}{Lina María García Hoyos}{\href{https://www.unbosque.edu.co/}{Universidad El Bosque}, Colombia}{2016 - 2018}{\begin{cvitems}
\item Trabajo de grado: \textit{¿Se puede determinar si una persona ha sido infiel a partir de su voz?}
\end{cvitems}}
    \cventry{Psicología}{Angie Liliana Pérez Rodríguez}{\href{https://www.unbosque.edu.co/}{Universidad El Bosque}, Colombia}{2016 - 2017}{\begin{cvitems}
\item Trabajo de grado: \textit{Efectos de los niveles hormonales en la discriminación tonal de mujeres}
\end{cvitems}}
    \cventry{Psicología}{Lina María Morales Sánchez}{\href{https://www.unbosque.edu.co/}{Universidad El Bosque}, Colombia}{2016 - 2017}{\begin{cvitems}
\item Trabajo de grado: \textit{Discriminación tonal predice satisfacción con pareja y su inversión parental, en hombres y mujeres}
\end{cvitems}}
    \cventry{Psicología}{Haydn Ricardo Roldán Morales}{\href{https://www.unbosque.edu.co/}{Universidad El Bosque}, Colombia}{2015 - 2016}{\begin{cvitems}
\item Trabajo de grado: \textit{El sonido de la seducción: discriminación tonal, satisfacción en pareja, y niveles hormonales en hombres}
\end{cvitems}}
\end{cventries}



\end{document}
