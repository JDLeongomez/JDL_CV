%!TEX TS-program = xelatex
%!TEX encoding = UTF-8 Unicode
% Awesome CV LaTeX Template for CV/Resume
%
% This template has been downloaded from:
% https://github.com/posquit0/Awesome-CV
%
% Author:
% Claud D. Park <posquit0.bj@gmail.com>
% http://www.posquit0.com
%
%
% Adapted to be an Rmarkdown template by Mitchell O'Hara-Wild
% 23 November 2018
%
% Template license:
% CC BY-SA 4.0 (https://creativecommons.org/licenses/by-sa/4.0/)
%
%-------------------------------------------------------------------------------
% CONFIGURATIONS
%-------------------------------------------------------------------------------
% A4 paper size by default, use 'letterpaper' for US letter
\documentclass[11pt,a4paper,]{awesome-cv}

% Configure page margins with geometry
\usepackage{geometry}
\geometry{left=1.4cm, top=.8cm, right=1.4cm, bottom=1.8cm, footskip=.5cm}


% Specify the location of the included fonts
\fontdir[fonts/]

% Color for highlights
% Awesome Colors: awesome-emerald, awesome-skyblue, awesome-red, awesome-pink, awesome-orange
%                 awesome-nephritis, awesome-concrete, awesome-darknight

\definecolor{awesome}{HTML}{333333}

% Colors for text
% Uncomment if you would like to specify your own color
% \definecolor{darktext}{HTML}{414141}
% \definecolor{text}{HTML}{333333}
% \definecolor{graytext}{HTML}{5D5D5D}
% \definecolor{lighttext}{HTML}{999999}

% Set false if you don't want to highlight section with awesome color
\setbool{acvSectionColorHighlight}{true}

% If you would like to change the social information separator from a pipe (|) to something else
\renewcommand{\acvHeaderSocialSep}{\quad\textbar\quad}

\def\endfirstpage{\newpage}

%-------------------------------------------------------------------------------
%	PERSONAL INFORMATION
%	Comment any of the lines below if they are not required
%-------------------------------------------------------------------------------
% Available options: circle|rectangle,edge/noedge,left/right

\name{Juan David}{Leongómez Peña}

\position{Profesor Asociado}
\address{PhD University of Stirling · MSc University of Liverpool}

\mobile{(+57) 319 407 7102}
\email{\href{mailto:juanleongomez@gmail.com}{\nolinkurl{juanleongomez@gmail.com}}}
\homepage{jdleongomez.info}

% \gitlab{gitlab-id}
% \stackoverflow{SO-id}{SO-name}
% \skype{skype-id}
% \reddit{reddit-id}


\usepackage{booktabs}

\providecommand{\tightlist}{%
	\setlength{\itemsep}{0pt}\setlength{\parskip}{0pt}}

%------------------------------------------------------------------------------


\usepackage{multicol} \usepackage{colortbl}

% Pandoc CSL macros
\newlength{\cslhangindent}
\setlength{\cslhangindent}{1.5em}
\newlength{\csllabelwidth}
\setlength{\csllabelwidth}{2em}
\newenvironment{CSLReferences}[2] % #1 hanging-ident, #2 entry spacing
 {% don't indent paragraphs
  \setlength{\parindent}{0pt}
  % turn on hanging indent if param 1 is 1
  \ifodd #1 \everypar{\setlength{\hangindent}{\cslhangindent}}\ignorespaces\fi
  % set entry spacing
  \ifnum #2 > 0
  \setlength{\parskip}{#2\baselineskip}
  \fi
 }%
 {}
\usepackage{calc}
\newcommand{\CSLBlock}[1]{#1\hfill\break}
\newcommand{\CSLLeftMargin}[1]{\parbox[t]{\csllabelwidth}{\honortitlestyle{#1}}}
\newcommand{\CSLRightInline}[1]{\parbox[t]{\linewidth - \csllabelwidth}{\honordatestyle{#1}}}
\newcommand{\CSLIndent}[1]{\hspace{\cslhangindent}#1}

\begin{document}

% Print the header with above personal informations
% Give optional argument to change alignment(C: center, L: left, R: right)
\makecvheader

% Print the footer with 3 arguments(<left>, <center>, <right>)
% Leave any of these blank if they are not needed
% 2019-02-14 Chris Umphlett - add flexibility to the document name in footer, rather than have it be static Curriculum Vitae
\makecvfooter
  {12 de noviembre de 2023}
    {Juan David Leongómez Peña~~~·~~~Cursos y su justificación}
  {\thepage}


%-------------------------------------------------------------------------------
%	CV/RESUME CONTENT
%	Each section is imported separately, open each file in turn to modify content
%------------------------------------------------------------------------------



\vspace{4mm}
\begin{tcolorbox}[enhanced,
        on line, 
        boxsep=4pt, left=0pt,right=0pt,top=0pt,bottom=0pt,
        colframe=white,colback=black]
  
\color{white}
\begin{LARGE}\begin{center}
Documento 2. \textbf{ Cursos y su justificación}
\end{center}\end{LARGE}
\end{tcolorbox}

\url{https://catalogo2.uniandes.edu.co/Catalogo_General_2013/Facultades/Facultad_de_Ciencias_Sociales/Departamento_de_Psicologia/listado_cursos.php}

\begin{table}[!h]
\centering\begingroup\fontsize{9}{11}\selectfont

\begin{tabular}{|>{\centering\arraybackslash}p{7em}|>{\centering\arraybackslash}p{6em}|>{\raggedright\arraybackslash}p{40em}|}
\hline
\em{\textbf{Curso}} & \em{\textbf{Nivel}} & \em{\textbf{Justificación (a partir de su experiencia o formación académica)}}\\
\hline
\vfill \vfill Proceso Básico XX (Lenguaje) & \vfill \vfill Pregrado & \textit{\textbf{Experiencia}}: \textcolor{red}{El propósito del curso es presentar al estudiante una panorámica básica de las problemáticas relacionadas con el lenguaje y, en particular, con su estructura, funcionamiento, uso producción, comprensión y adquisición. Se abordarán tanto estudios clásicos como desarrollos recientes en el área de la lingüística y la psicolingüística.}\\
 &  & \textit{\textbf{Formación}}: Maestría en Metodología de la Investigación, Universidad de Liverpool, Inglaterra.\\
\hline
\vfill \vfill Diseño y análisis
I & \vfill \vfill Pregrado & \textit{\textbf{Experiencia}}: 12 años de experiencia como investigador, asesor y docente a nivel de pregrado y posgrado de asignaturas relacionadas con métodos cuantitativos, tanto en Colombia como en el Reino Unido. Dirección y codirección de múltiples trabajos de grado, incluyendo dos trabajos doctorales, y experiencia en el análisis y modelamiento estadístico de estudios como se puede ver reflejado la mayoría de mis publicaciones. Creación de materiales educativos y didácticos (guías, videos y \textit{apps}) para la enseñanza de conceptos estadísticos y la adopción de buenas prácticas de investigación cuantitativa.\\
 &  & \textit{\textbf{Formación}}: Maestría (Universidad de Liverpool, Inglaterra) y Doctorado (Universidad de Stirling, Escocia) con énfasis en métodos de \vphantom{3} investigación.\\
\hline
\vfill \vfill Diseño y análisis
II & \vfill \vfill Pregrado & \textit{\textbf{Experiencia}}: 12 años de experiencia como investigador, asesor y docente a nivel de pregrado y posgrado de asignaturas relacionadas con métodos cuantitativos, tanto en Colombia como en el Reino Unido. Dirección y codirección de múltiples trabajos de grado, incluyendo dos trabajos doctorales, y experiencia en el análisis y modelamiento estadístico de estudios como se puede ver reflejado la mayoría de mis publicaciones. Creación de materiales educativos y didácticos (guías, videos y \textit{apps}) para la enseñanza de conceptos estadísticos y la adopción de buenas prácticas de investigación cuantitativa.\\
 &  & \textit{\textbf{Formación}}: Maestría (Universidad de Liverpool, Inglaterra) y Doctorado (Universidad de Stirling, Escocia) con énfasis en métodos de \vphantom{2} investigación.\\
\hline
\vfill \vfill Métodos cuantitativos avanzados & \vfill \vfill Posgrado (Maestría en Psicología) & \textit{\textbf{Experiencia}}: 12 años de experiencia como investigador, asesor y docente a nivel de pregrado y posgrado de asignaturas relacionadas con métodos cuantitativos, tanto en Colombia como en el Reino Unido. Dirección y codirección de múltiples trabajos de grado, incluyendo dos trabajos doctorales, y experiencia en el análisis y modelamiento estadístico de estudios como se puede ver reflejado la mayoría de mis publicaciones. Creación de materiales educativos y didácticos (guías, videos y \textit{apps}) para la enseñanza de conceptos estadísticos y la adopción de buenas prácticas de investigación cuantitativa.\\
 &  & \textit{\textbf{Formación}}: Maestría (Universidad de Liverpool, Inglaterra) y Doctorado (Universidad de Stirling, Escocia) con énfasis en métodos de \vphantom{1} investigación.\\
\hline
\vfill \vfill Curso Metodológico 1 & \vfill \vfill Posgrado (Doctorado) & \textit{\textbf{Experiencia}}: 12 años de experiencia como investigador, asesor y docente a nivel de pregrado y posgrado de asignaturas relacionadas con métodos cuantitativos, tanto en Colombia como en el Reino Unido. Dirección y codirección de múltiples trabajos de grado, incluyendo dos trabajos doctorales, y experiencia en el análisis y modelamiento estadístico de estudios como se puede ver reflejado la mayoría de mis publicaciones. Creación de materiales educativos y didácticos (guías, videos y \textit{apps}) para la enseñanza de conceptos estadísticos y la adopción de buenas prácticas de investigación cuantitativa.\\
 &  & \textit{\textbf{Formación}}: Maestría (Universidad de Liverpool, Inglaterra) y Doctorado (Universidad de Stirling, Escocia) con énfasis en métodos de investigación.\\
\hline
\cellcolor{gray}{\textcolor{white}{\vfill \vfill Bases evolutivas del comportamiento}} & \cellcolor{gray}{\textcolor{white}{\vfill \vfill Electiva}} & \cellcolor{gray}{\textcolor{white}{\textit{\textbf{Experiencia}}: El propósito del curso es presentar al estudiante una panorámica básica de las problemáticas relacionadas con el lenguaje y, en particular, con su estructura, funcionamiento, uso producción, comprensión y adquisición. Se abordarán tanto estudios clásicos como desarrollos recientes en el área de la lingüística y la psicolingüística.}}\\
\hline
\cellcolor{gray}{\textcolor{white}{\vfill \vfill Evolución cognitiva}} & \cellcolor{gray}{\textcolor{white}{\vfill \vfill Electiva}} & \cellcolor{gray}{\textcolor{white}{\textit{\textbf{Experiencia}}: El propósito del curso es presentar al estudiante una panorámica básica de las problemáticas relacionadas con el lenguaje y, en particular, con su estructura, funcionamiento, uso producción, comprensión y adquisición. Se abordarán tanto estudios clásicos como desarrollos recientes en el área de la lingüística y la psicolingüística.}}\\
\hline
\cellcolor{black}{\textcolor{white}{\vfill \vfill Métodos Reproducibles y Ciencia Abierta}} & \cellcolor{black}{\textcolor{white}{\vfill \vfill Educación Continua}} & \cellcolor{black}{\textcolor{white}{\textit{\textbf{Experiencia}}: El propósito del curso es presentar al estudiante una panorámica básica de las problemáticas relacionadas con el lenguaje y, en particular, con su estructura, funcionamiento, uso producción, comprensión y adquisición. Se abordarán tanto estudios clásicos como desarrollos recientes en el área de la lingüística y la psicolingüística.}}\\
\hline
\end{tabular}
\endgroup{}
\end{table}



\end{document}
