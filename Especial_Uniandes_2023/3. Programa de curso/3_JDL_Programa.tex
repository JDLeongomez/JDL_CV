%!TEX TS-program = xelatex
%!TEX encoding = UTF-8 Unicode
% Awesome CV LaTeX Template for CV/Resume
%
% This template has been downloaded from:
% https://github.com/posquit0/Awesome-CV
%
% Author:
% Claud D. Park <posquit0.bj@gmail.com>
% http://www.posquit0.com
%
%
% Adapted to be an Rmarkdown template by Mitchell O'Hara-Wild
% 23 November 2018
%
% Template license:
% CC BY-SA 4.0 (https://creativecommons.org/licenses/by-sa/4.0/)
%
%-------------------------------------------------------------------------------
% CONFIGURATIONS
%-------------------------------------------------------------------------------
% A4 paper size by default, use 'letterpaper' for US letter
\documentclass[11pt,a4paper,]{awesome-cv}

% Configure page margins with geometry
\usepackage{geometry}
\geometry{left=1.4cm, top=.8cm, right=1.4cm, bottom=1.8cm, footskip=.5cm}


% Specify the location of the included fonts
\fontdir[fonts/]

% Color for highlights
% Awesome Colors: awesome-emerald, awesome-skyblue, awesome-red, awesome-pink, awesome-orange
%                 awesome-nephritis, awesome-concrete, awesome-darknight

\definecolor{awesome}{HTML}{333333}

% Colors for text
% Uncomment if you would like to specify your own color
% \definecolor{darktext}{HTML}{414141}
% \definecolor{text}{HTML}{333333}
% \definecolor{graytext}{HTML}{5D5D5D}
% \definecolor{lighttext}{HTML}{999999}

% Set false if you don't want to highlight section with awesome color
\setbool{acvSectionColorHighlight}{true}

% If you would like to change the social information separator from a pipe (|) to something else
\renewcommand{\acvHeaderSocialSep}{\quad\textbar\quad}

\def\endfirstpage{\newpage}

%-------------------------------------------------------------------------------
%	PERSONAL INFORMATION
%	Comment any of the lines below if they are not required
%-------------------------------------------------------------------------------
% Available options: circle|rectangle,edge/noedge,left/right

\name{Juan David}{Leongómez Peña}

\position{Profesor Asociado}
\address{PhD University of Stirling · MSc University of Liverpool}

\mobile{(+57) 319 407 7102}
\email{\href{mailto:juanleongomez@gmail.com}{\nolinkurl{juanleongomez@gmail.com}}}
\homepage{jdleongomez.info}

% \gitlab{gitlab-id}
% \stackoverflow{SO-id}{SO-name}
% \skype{skype-id}
% \reddit{reddit-id}


\usepackage{booktabs}

\providecommand{\tightlist}{%
	\setlength{\itemsep}{0pt}\setlength{\parskip}{0pt}}

%------------------------------------------------------------------------------



% Pandoc CSL macros
\newlength{\cslhangindent}
\setlength{\cslhangindent}{1.5em}
\newlength{\csllabelwidth}
\setlength{\csllabelwidth}{2em}
\newenvironment{CSLReferences}[2] % #1 hanging-ident, #2 entry spacing
 {% don't indent paragraphs
  \setlength{\parindent}{0pt}
  % turn on hanging indent if param 1 is 1
  \ifodd #1 \everypar{\setlength{\hangindent}{\cslhangindent}}\ignorespaces\fi
  % set entry spacing
  \ifnum #2 > 0
  \setlength{\parskip}{#2\baselineskip}
  \fi
 }%
 {}
\usepackage{calc}
\newcommand{\CSLBlock}[1]{#1\hfill\break}
\newcommand{\CSLLeftMargin}[1]{\parbox[t]{\csllabelwidth}{\honortitlestyle{#1}}}
\newcommand{\CSLRightInline}[1]{\parbox[t]{\linewidth - \csllabelwidth}{\honordatestyle{#1}}}
\newcommand{\CSLIndent}[1]{\hspace{\cslhangindent}#1}

\begin{document}

% Print the header with above personal informations
% Give optional argument to change alignment(C: center, L: left, R: right)
\makecvheader

% Print the footer with 3 arguments(<left>, <center>, <right>)
% Leave any of these blank if they are not needed
% 2019-02-14 Chris Umphlett - add flexibility to the document name in footer, rather than have it be static Curriculum Vitae
\makecvfooter
  {14 de noviembre de 2023}
    {Juan David Leongómez Peña~~~·~~~Programa de curso}
  {\thepage}


%-------------------------------------------------------------------------------
%	CV/RESUME CONTENT
%	Each section is imported separately, open each file in turn to modify content
%------------------------------------------------------------------------------



\vspace{4mm}
\begin{tcolorbox}[enhanced,
        on line, 
        boxsep=4pt, left=0pt,right=0pt,top=0pt,bottom=0pt,
        colframe=white,colback=black]
  
\color{white}
\begin{LARGE}\begin{center}
Documento 3. \textbf{Programa de curso}
\end{center}\end{LARGE}
\end{tcolorbox}

\begin{LARGE}\begin{center}
\textbf{Proceso Básico: Lenguaje}\linebreak
Departamento de Psicología
\end{center}\end{LARGE}

\begin{cvskills}
  \cvskill
    {TIPO DE CURSO}
    {\textbf{Seminario}}
    
  \cvskill
    {PROFESOR}
    {Juan David Leongómez}
\end{cvskills}

\hypertarget{descripciuxf3n-general}{%
\section{Descripción General}\label{descripciuxf3n-general}}

¿Cómo aprenden los bebés a hablar? ¿Por qué somos el único animal que
habla? Y ¿por qué somos el único animal realmente musical? Estas son
preguntas complejas relacionadas con la comunicación vocal, y tema
central de la investigación reciente. Responderlas revelaría aspectos
fundamentales del ser humano. Nuestra comunicación vocal, y en especial
nuestro lenguaje y música han, después de todo, moldeado profundamente
nuestra biología, nuestra psicología, y cada aspecto de nuestra cultura.

Existen fascinantes ejemplos de precursores animales que tienen
similitudes con la comunicación vocal en humanos, incluyendo casos de
mensajes acústicos complejos, e incluso proto-sintaxis y
proto-semántica; sin embargo, ninguna especie llega a un nivel de
complejidad similar al de la música y lenguaje humanos.

La idea de éste curso, en formato de seminario, es analizar los
diferentes aspectos de la comunicación vocal y la literatura reciente,
haciendo énfasis en los métodos utilizados para el análisis vocal
acústico, así como el análisis del significado de las vocalizaciones, a
partir de tres aspectos fundamentales: (1) análisis comparativos entre
humanos y otras especies (principalmente mamíferos y aves), (2)
evolución de la comunicación vocal, y (3) desarrollo de habilidades
cognoscitivas y sensoperceptuales para la música y el lenguaje.

Quienes participen obtendrán un conocimiento general de la literatura en
éste campo, incluyendo la acústica de las voces, los mensajes tanto
biológicos como psicológicos que transmiten y su relevancia social, las
principales técnicas para su estudio, y las principales teorías sobre el
origen del lenguaje y la música.

El seminario es ideal para cualquier persona con un interés general en
el tema o, por ejemplo, para cualquiera que se interese el origen y
desarrollo del lenguaje, en cómo los bebés aprenden a hablar, en la
misteriosa musicalidad humana, en la comunicación animal, o para quien
quiera comprender la mecánica de la comunicación vocal y los mensajes
que transmitimos constantemente de manera consciente e inconsciente.

\hypertarget{resultados-de-aprendizaje}{%
\section{Resultados de Aprendizaje}\label{resultados-de-aprendizaje}}

Al finalizar el curso, los estudiantes estarán en capacidad de:

\begin{enumerate}
\def\labelenumi{\arabic{enumi}.}
\tightlist
\item
  Analizar los aportes de diversas disciplinas en el estudio de la
  psicología del lenguaje
\item
  Sintetizar y evaluar los aspectos principales de la literatura
  relevante
\item
  Afrontar la literatura de manera crítica
\item
  Aportar ideas con cierto nivel de originalidad, a partir de la
  relación entre las conclusiones de la literatura relevante para el
  curso
\item
  Comparar los límites y los alcances de las aproximaciones
  metodológicas a la investigación sobre el proceso
\end{enumerate}

\hypertarget{metodologuxeda}{%
\section{Metodología}\label{metodologuxeda}}

Este curso se basará en una versión simplificada del método de
\textit{seminario alemán}, adaptada para el nivel de formación.

Bajo la dirección del profesor, se abordarán temáticas específicas en
cada sesión, a partir de la lectura de artículos relevantes.

Aunque todas y todos los estudiantes deberán hacer una lectura general
de los artículos, para cada sesión algunos o algunas estudiantes deberán
leer uno de los artículos a profundidad, apoyándose en asesorías del
profesor, y preparar tanto (1) una exposición del artículo, como (2) una
actividad ilustrativa y didáctica sobre sus contenidos (por ejemplo, un
\textit{pub quiz}, o un concurso de interpretación de sus resultados).

Al terminar las exposiciones, se discutirán los resultados e
implicaciones del tema de cada sesión, bajo la guía del profesor, quien
buscará señalar limitaciones, fortalezas y explicaciones alternativas,
profundizando tanto en las bases teóricas, como en las fortalezas y
limitaciones analíticas y metodológicas de cada artículo, con el
objetivo de fomentar la lectura crítica.

\hypertarget{evaluaciuxf3n}{%
\section{Evaluación}\label{evaluaciuxf3n}}

Como seminario, la clase se basa en la presentación y discusión de
artículos importantes de la literatura. Se evalúa tanto la presentación
(síntesis) de artículos y las actividades ilustrativas y didácticas
sobre sus contenidos, como las intervenciones en la subsecuente
discusión.

Ensayos sobre preguntas específicas, que vinculan elementos de varios
artículos artículos y disciplinas sobre los temas tratados, que
permitirán permiten que cada estudiante presente sus propias relaciones
y conclusiones, así como buscar literatura adicional.

\hypertarget{estructura-del-curso}{%
\section{Estructura del Curso}\label{estructura-del-curso}}

\begin{table}[!h]
\centering\begingroup\fontsize{7.5}{9.5}\selectfont

\begin{tabular}{|>{\centering\arraybackslash}p{4em}|>{\raggedright\arraybackslash}p{28em}|>{\raggedright\arraybackslash}p{12em}|>{\raggedright\arraybackslash}p{14em}|}
\hline
\multicolumn{1}{>{\centering\arraybackslash}p{4em}}{\begingroup\fontsize{11}{13}\selectfont \em{\textbf{Semana}}\endgroup} & \multicolumn{1}{>{\centering\arraybackslash}p{28em}}{\begingroup\fontsize{11}{13}\selectfont \em{\textbf{Tema}}\endgroup} & \multicolumn{1}{>{\centering\arraybackslash}p{12em}}{\begingroup\fontsize{11}{13}\selectfont \em{\textbf{Lecturas}}\endgroup} & \multicolumn{1}{>{\centering\arraybackslash}p{14em}}{\begingroup\fontsize{11}{13}\selectfont \em{\textbf{Trabajo autónomo}}\endgroup}\\
\hline
1 & Introducción y presentación del curso.\linebreak Evolución y significado de las señales vocales. & — & —\\
\hline
2 & Percepción a partir de la voz en animales: tamaño corporal, dominancia, masculinidad-feminidad y selección sexual & \cite{beeMaleGreenFrogs2000}\linebreak \cite{charltonContextrelatedAcousticVariation2011}\linebreak \cite{RefWorks:723} & Lectura de artículos\linebreak (Preparación exposición y actividad)\\
\hline
3 & Percepción a partir de la voz en humanos: ¿Qué explica el fuerte dimorfismo sexual de las voces humanas? & \cite{Collins2000}\linebreak \cite{collinsVocalVisualAttractiveness2003}\linebreak \cite{putsDominanceEvolutionSexual2006}\linebreak \cite{putsSexualSelectionMale2016} & Lectura de artículos\linebreak (Preparación exposición y actividad)\\
\hline
4 & Percepciones de dominancia en humanos & \cite{RefWorks:452}\linebreak \cite{leongomezPerceivedDifferencesSocial2017}\linebreak \cite{kleisnerPredictingStrengthAggressive2021}\linebreak \cite{pisanskiReturnOzVoice2014} & Lectura de artículos\linebreak (Preparación exposición y actividad)\\
\hline
5 & Percepción de atractivo físico en humanos: hormonas, masculinidad y feminidad & \cite{feinbergManipulationsFundamentalFormant2005}\linebreak \cite{RefWorks:385}\linebreak \cite{leongomezVocalModulationCourtship2014} & Lectura de artículos\linebreak (Preparación exposición y actividad)\\
\hline
6 & ENSAYO ¿Qué señales contiene la voz humana? & — & Preparación ensayo\\
\hline
7 & Precursores animales del lenguaje: proto-semántica & \cite{evansChickenFoodCalls1999}\linebreak \cite{greeneRedSquirrelsTamiasciurus1998}\linebreak \cite{seyfarthMonkeyResponsesThree1980} & Lectura de artículos\linebreak (Preparación exposición y actividad)\\
\hline
8 & Precursores animales del lenguaje: proto-sintaxis & \cite{marlerSpeciesuniversalMicrostructureLearned1984}\linebreak \cite{podosPermissivenessLearningDevelopment1999}\linebreak \cite{zuberbuhlerSyntaxCompositionalityAnimal2019} & Lectura de artículos\linebreak (Preparación exposición y actividad)\\
\hline
9 & Teorías del origen del lenguaje: expresiones faciales, acicalamiento social & \cite{dunbarOriginSubsequentEvolution2003}\linebreak \cite{dunbarCoevolutionNeocorticalSize1993a}\linebreak \cite{mccombCoevolutionVocalCommunication2005} & Lectura de artículos\linebreak (Preparación exposición y actividad)\\
\hline
10 & Evolución cultural de la comunicación vocal: casos animales & \cite{eriksenCulturalChangeSongs2005}\linebreak \cite{lutherUrbanNoiseCultural2010}\linebreak \cite{noadCulturalRevolutionWhale2000} & Lectura de artículos\linebreak (Preparación exposición y actividad)\\
\hline
11 & Discusión general: Ideas sobre el origen del lenguaje & \cite{RefWorks:428}\linebreak \cite{RefWorks:463}\linebreak \cite{balterEvolutionLanguageAnimal2010}\linebreak \cite{fitchBiologyEvolutionSpeech2018} & Lectura de artículos\linebreak (Preparación exposición y actividad)\\
\hline
12 & ENSAYO ¿Qué diferencias y similitudes existen entre el lenguaje humano y la comunicación vocal en animales? & — & Preparación ensayo\\
\hline
13 & Similitud entre música y lenguaje: evidencia de recursos compartidos & \cite{sammlerOverlapMusicalLinguistic2009}\linebreak \cite{koelschAdultsChildrenProcessing2005}\linebreak \cite{coumelSecondLanguageAccent2019}\linebreak \cite{zuberbuhlerSyntaxCompositionalityAnimal2019} & Lectura de artículos\linebreak (Preparación exposición y actividad)\\
\hline
14 & Relación entre música y lenguaje: evidencia en daños cerebrales y desórdenes del desarrollo & \cite{jentschkeChildrenSpecificLanguage2008}\linebreak \cite{pearceSelectedObservationsAmusia2005}\linebreak \cite{signoretAphasiaAmusiaBlind1987} & Lectura de artículos\linebreak (Preparación exposición y actividad)\\
\hline
15 & Maternés: comunicación emocional & \cite{falkPrelinguisticEvolutionEarly2005}\linebreak \cite{kemlernelsonHowProsodicCues2009}\linebreak \cite{papousekMeaningsMelodiesMotherese1991} & Lectura de artículos\linebreak (Preparación exposición y actividad)\\
\hline
16 & Evolución de la musicalidad & \cite{fitchBiologyEvolutionMusic2006a}\linebreak \cite{mehrOriginsMusicCredible2021}\linebreak \cite{savageMusicCoevolvedSystem2021}\linebreak \cite{leongomezMusicalityHumanVocal2022} & Lectura de artículos\linebreak (Preparación exposición y actividad)\\
\hline
\end{tabular}
\endgroup{}
\end{table}

\hypertarget{referencias}{%
\section{Referencias}\label{referencias}}

\begin{multicols}{2}
\AtNextBibliography{\footnotesize}
\printbibliography[heading=none]
\end{multicols}



\end{document}
