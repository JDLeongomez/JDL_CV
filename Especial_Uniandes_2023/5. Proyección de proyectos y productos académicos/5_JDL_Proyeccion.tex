%!TEX TS-program = xelatex
%!TEX encoding = UTF-8 Unicode
% Awesome CV LaTeX Template for CV/Resume
%
% This template has been downloaded from:
% https://github.com/posquit0/Awesome-CV
%
% Author:
% Claud D. Park <posquit0.bj@gmail.com>
% http://www.posquit0.com
%
%
% Adapted to be an Rmarkdown template by Mitchell O'Hara-Wild
% 23 November 2018
%
% Template license:
% CC BY-SA 4.0 (https://creativecommons.org/licenses/by-sa/4.0/)
%
%-------------------------------------------------------------------------------
% CONFIGURATIONS
%-------------------------------------------------------------------------------
% A4 paper size by default, use 'letterpaper' for US letter
\documentclass[11pt,a4paper,]{awesome-cv}

% Configure page margins with geometry
\usepackage{geometry}
\geometry{left=1.4cm, top=.8cm, right=1.4cm, bottom=1.8cm, footskip=.5cm}


% Specify the location of the included fonts
\fontdir[fonts/]

% Color for highlights
% Awesome Colors: awesome-emerald, awesome-skyblue, awesome-red, awesome-pink, awesome-orange
%                 awesome-nephritis, awesome-concrete, awesome-darknight

\definecolor{awesome}{HTML}{333333}

% Colors for text
% Uncomment if you would like to specify your own color
% \definecolor{darktext}{HTML}{414141}
% \definecolor{text}{HTML}{333333}
% \definecolor{graytext}{HTML}{5D5D5D}
% \definecolor{lighttext}{HTML}{999999}

% Set false if you don't want to highlight section with awesome color
\setbool{acvSectionColorHighlight}{true}

% If you would like to change the social information separator from a pipe (|) to something else
\renewcommand{\acvHeaderSocialSep}{\quad\textbar\quad}

\def\endfirstpage{\newpage}

%-------------------------------------------------------------------------------
%	PERSONAL INFORMATION
%	Comment any of the lines below if they are not required
%-------------------------------------------------------------------------------
% Available options: circle|rectangle,edge/noedge,left/right

\name{Juan David}{Leongómez Peña}

\position{Profesor Asociado}
\address{PhD University of Stirling · MSc University of Liverpool}

\mobile{(+57) 319 407 7102}
\email{\href{mailto:juanleongomez@gmail.com}{\nolinkurl{juanleongomez@gmail.com}}}
\homepage{jdleongomez.info}

% \gitlab{gitlab-id}
% \stackoverflow{SO-id}{SO-name}
% \skype{skype-id}
% \reddit{reddit-id}


\usepackage{booktabs}

\providecommand{\tightlist}{%
	\setlength{\itemsep}{0pt}\setlength{\parskip}{0pt}}

%------------------------------------------------------------------------------


\usepackage{multicol} \usepackage{colortbl}

% Pandoc CSL macros
\newlength{\cslhangindent}
\setlength{\cslhangindent}{1.5em}
\newlength{\csllabelwidth}
\setlength{\csllabelwidth}{2em}
\newenvironment{CSLReferences}[2] % #1 hanging-ident, #2 entry spacing
 {% don't indent paragraphs
  \setlength{\parindent}{0pt}
  % turn on hanging indent if param 1 is 1
  \ifodd #1 \everypar{\setlength{\hangindent}{\cslhangindent}}\ignorespaces\fi
  % set entry spacing
  \ifnum #2 > 0
  \setlength{\parskip}{#2\baselineskip}
  \fi
 }%
 {}
\usepackage{calc}
\newcommand{\CSLBlock}[1]{#1\hfill\break}
\newcommand{\CSLLeftMargin}[1]{\parbox[t]{\csllabelwidth}{\honortitlestyle{#1}}}
\newcommand{\CSLRightInline}[1]{\parbox[t]{\linewidth - \csllabelwidth}{\honordatestyle{#1}}}
\newcommand{\CSLIndent}[1]{\hspace{\cslhangindent}#1}

\begin{document}

% Print the header with above personal informations
% Give optional argument to change alignment(C: center, L: left, R: right)
\makecvheader

% Print the footer with 3 arguments(<left>, <center>, <right>)
% Leave any of these blank if they are not needed
% 2019-02-14 Chris Umphlett - add flexibility to the document name in footer, rather than have it be static Curriculum Vitae
\makecvfooter
  {19 de noviembre de 2023}
    {Juan David Leongómez Peña~~~·~~~Proyección de proyectos y productos
académicos}
  {\thepage}


%-------------------------------------------------------------------------------
%	CV/RESUME CONTENT
%	Each section is imported separately, open each file in turn to modify content
%------------------------------------------------------------------------------



\vspace{4mm}
\begin{tcolorbox}[enhanced,
        on line, 
        boxsep=4pt, left=0pt,right=0pt,top=0pt,bottom=0pt,
        colframe=white,colback=black]
  
\color{white}
\begin{LARGE}\begin{center}
Documento 5. \textbf{Proyección de proyectos y productos académicos}
\end{center}\end{LARGE}
\end{tcolorbox}

\hypertarget{luxednea-efectos-sociales-y-percepciuxf3n-interpersonal-a-partir-de-seuxf1ales-vocales}{%
\section{Línea efectos sociales y percepción interpersonal a partir de
señales
vocales}\label{luxednea-efectos-sociales-y-percepciuxf3n-interpersonal-a-partir-de-seuxf1ales-vocales}}

\space

\hypertarget{seuxf1ales-de-infidelidad-y-sociosexualidad-en-rostros-y-voces-rol-investigador-principal}{%
\subsection{\texorpdfstring{Señales de infidelidad y sociosexualidad en
rostros y voces (Rol: \emph{investigador
principal})}{Señales de infidelidad y sociosexualidad en rostros y voces (Rol: investigador principal)}}\label{seuxf1ales-de-infidelidad-y-sociosexualidad-en-rostros-y-voces-rol-investigador-principal}}

\begin{footnotesize}

Espero publicar 1 artículo Q1 (año uno) con los análisis principales de este estudio, combinando señales vocales y faciales. Adicionalmente, espero publicar un segundo estudio Q1 (año 3) estudiando la percepción multimodal de sociosexualidad.

\end{footnotesize}

\hypertarget{modulaciuxf3n-vocal-especificidad-contextual-y-efectos-sociales-rol-investigador-principal}{%
\subsection{\texorpdfstring{Modulación Vocal: Especificidad Contextual y
Efectos Sociales (Rol: \emph{investigador
principal})}{Modulación Vocal: Especificidad Contextual y Efectos Sociales (Rol: investigador principal)}}\label{modulaciuxf3n-vocal-especificidad-contextual-y-efectos-sociales-rol-investigador-principal}}

\begin{footnotesize}

Este proyecto, planeado y liderado por mí como investigador principal, reúne a un grupo de importantes investigadoras e investigadores al rededor del mundo, incluyendo, entre otras personas, al Dr. \href{https://scholar.google.com/citations?user=fZ_ZqrMAAAAJ}{David Puts} (Department of Anthropology, Pennsylvania State University, EEUU), la Dra. \href{https://scholar.google.com/citations?user=sTCqMrwAAAAJ}{Katarzyna Pisanski} (Sensory Neuro-Ethology Lab (ENES), University of Lyon/Saint-Etienne, Francia), el Dr. \href{https://scholar.google.com/citations?user=6jMFwJQAAAAJ}{Pablo Arias Sarah} (School of Neuroscience and Psychology, University of Glasgow, Reino Unido),  la Dra. \href{https://scholar.google.cz/citations?user=BaIq_QIAAAAJ}{Jaroslava Varella} Valentova (Instituto de Psicología, Universidad de São Paulo, Brasil), el Dr. \href{https://scholar.google.com/citations?user=ZtzwhVsAAAAJ}{Karel Kleisner} (Department of Philosophy and History of Science, Charles University, República Checa), y el Dr. \href{https://scholar.google.com/citations?user=nhsFlxAAAAAJ}{Isaac G-Santoyo} (Facultad de Psicología, Universidad Nacional Autónoma de México). 

El proyecto consta de seis estudios, pero para ser realizado por completo requiere de financiación (la propuesta ya fue sometida a un \textit{grant}, y está a la espera de respuesta). Asumiendo que el proyecto obtenga financiación, espero publicar al menos 4 artículos de investigación Q1, y al menos 2 en revistas generalistas de primer nivel (e.g., Nature Human Behaviour). El proyecto contempla que cada artículo sea publicado siguiendo el modelo de \href{https://www.nature.com/articles/s41562-021-01193-7}{\textit{reporte registrado}}. De no obtener financiación para el propyecto completo, se buscarán fondos para comenzar haciendo estudios específicos, y la producción se ajustará de acuerdo a esto.

\end{footnotesize}

\hypertarget{colaboraciones}{%
\section{Colaboraciones}\label{colaboraciones}}

\hypertarget{disgust-perception-through-sounds-and-sights-rol-co-investigador}{%
\subsection{\texorpdfstring{Disgust Perception through Sounds and Sights
(Rol:
\emph{co-investigador})}{Disgust Perception through Sounds and Sights (Rol: co-investigador)}}\label{disgust-perception-through-sounds-and-sights-rol-co-investigador}}

\begin{footnotesize}

Espero publicar al menos 1 artículo Q1 (año uno) con los resultados de este estudio global.

\end{footnotesize}

\hypertarget{efecto-de-seuxf1ales-estuxe1ticas-evolutivamente-relevantes-sexo-dominancia-y-atractivo-en-el-procesamiento-cortical-de-rostros-humanos-rol-co-investigador}{%
\subsection{\texorpdfstring{Efecto de señales estáticas evolutivamente
relevantes (sexo, dominancia y atractivo) en el procesamiento cortical
de rostros humanos (Rol:
\emph{co-investigador})}{Efecto de señales estáticas evolutivamente relevantes (sexo, dominancia y atractivo) en el procesamiento cortical de rostros humanos (Rol: co-investigador)}}\label{efecto-de-seuxf1ales-estuxe1ticas-evolutivamente-relevantes-sexo-dominancia-y-atractivo-en-el-procesamiento-cortical-de-rostros-humanos-rol-co-investigador}}

\begin{footnotesize}

Espero publicar un artículo Q1 (año dos) con los resultados de un estudio usando \textit{eye-tracking}.

\end{footnotesize}

\hypertarget{crosscultural-relationships-between-music-emotion-and-visual-imagery-a-comparative-study-of-iran-canada-and-japanrol-co-investigador}{%
\subsection{\texorpdfstring{Cross‑cultural relationships between music,
emotion, and visual imagery: A comparative study of Iran, Canada, and
Japan(Rol:
\emph{co-investigador})}{Cross‑cultural relationships between music, emotion, and visual imagery: A comparative study of Iran, Canada, and Japan(Rol: co-investigador)}}\label{crosscultural-relationships-between-music-emotion-and-visual-imagery-a-comparative-study-of-iran-canada-and-japanrol-co-investigador}}

\begin{footnotesize}

Espero que el \textit{Stage 1} de este reporte registrado, para el que hice un análisis de poder basado en simulaciones (disponible \href{https://github.com/comp-music-lab/VisualEars/blob/main/Power_Analysis/Power_analysis.pdf}{acá}), sea aceptado en principio (año uno), y el \textit{Stage 2} sea publicado en revista Q1 (año dos).

\end{footnotesize}

\hypertarget{efecto-del-control-de-los-recursos-real-y-simulado-sobre-las-preferencias-de-mujeres-andruxf3filas-por-la-masculinidad-en-rostros-de-hombres-un-estudio-experimental-usando-rastreo-ocular-rol-co-investigador}{%
\subsection{\texorpdfstring{Efecto del control de los recursos real y
simulado sobre las preferencias de mujeres andrófilas por la
masculinidad en rostros de hombres: un estudio experimental usando
rastreo ocular (Rol:
\emph{co-investigador})}{Efecto del control de los recursos real y simulado sobre las preferencias de mujeres andrófilas por la masculinidad en rostros de hombres: un estudio experimental usando rastreo ocular (Rol: co-investigador)}}\label{efecto-del-control-de-los-recursos-real-y-simulado-sobre-las-preferencias-de-mujeres-andruxf3filas-por-la-masculinidad-en-rostros-de-hombres-un-estudio-experimental-usando-rastreo-ocular-rol-co-investigador}}

\begin{footnotesize}

Espero publicar 1 artículo Q1 (año tres) con los resultados de este estudio.

\end{footnotesize}

\hypertarget{productos-no-convencionales-y-divulgaciuxf3n-de-ciencia-abierta}{%
\section{Productos no convencionales y divulgación de Ciencia
Abierta}\label{productos-no-convencionales-y-divulgaciuxf3n-de-ciencia-abierta}}

\begin{footnotesize}

Espero publicar más aplicaciones web que ayuden a ilustrar conceptos fundamentales de estadística frecuentista a partir de simulaciones. Por ejemplo, como parte de la colección \href{https://shiny.jdl-svr.lat/PowerSimulate/}{\textit{PowerSimulate}}, planeo crear aplicaciones para realizar análisis de poder estadístico y calcular el tamaño de muestra para (1) ANOVA de una vía de medidas independientes, (2) ANOVA de una vía de medidas repetidas, y (3) ANOVA factorial (probablemente 2x2) de medidas independientes, (4) repetidas y (5) mixtas. Todas estas aplicaciones serían creadas y publicadas durante el año uno.

Adicionalmente, espero editar un libro de técnicas de análisis en R para investigadores e investigadoras en ciencias sociales, en el que cada capítulo sea una guía detallada, paso a paso y fácil de seguir, que permita a las y los lectores entender y hacer análisis (incluso análisis relativamente complejos), incluso si tienen muy poca experiencia usando R (año 3). Para maximizar su alcance y accesibilidad, espero que cada capítulo esté acompañado de guías concretas en video o medios interactivos como aplicaciones creadas usando \textit{Shiny}. Esta idea fue anteriormente discutida con profesores de la Facultad de Psicología de la Universidad de los Andes. 

\end{footnotesize}



\end{document}
