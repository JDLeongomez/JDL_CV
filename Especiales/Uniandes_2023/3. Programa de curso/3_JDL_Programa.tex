%!TEX TS-program = xelatex
%!TEX encoding = UTF-8 Unicode
% Awesome CV LaTeX Template for CV/Resume
%
% This template has been downloaded from:
% https://github.com/posquit0/Awesome-CV
%
% Author:
% Claud D. Park <posquit0.bj@gmail.com>
% http://www.posquit0.com
%
%
% Adapted to be an Rmarkdown template by Mitchell O'Hara-Wild
% 23 November 2018
%
% Template license:
% CC BY-SA 4.0 (https://creativecommons.org/licenses/by-sa/4.0/)
%
%-------------------------------------------------------------------------------
% CONFIGURATIONS
%-------------------------------------------------------------------------------
% A4 paper size by default, use 'letterpaper' for US letter
\documentclass[11pt,a4paper,]{awesome-cv}

% Configure page margins with geometry
\usepackage{geometry}
\geometry{left=1.4cm, top=.8cm, right=1.4cm, bottom=1.8cm, footskip=.5cm}


% Specify the location of the included fonts
\fontdir[fonts/]

% Color for highlights
% Awesome Colors: awesome-emerald, awesome-skyblue, awesome-red, awesome-pink, awesome-orange
%                 awesome-nephritis, awesome-concrete, awesome-darknight

\definecolor{awesome}{HTML}{333333}

% Colors for text
% Uncomment if you would like to specify your own color
% \definecolor{darktext}{HTML}{414141}
% \definecolor{text}{HTML}{333333}
% \definecolor{graytext}{HTML}{5D5D5D}
% \definecolor{lighttext}{HTML}{999999}

% Set false if you don't want to highlight section with awesome color
\setbool{acvSectionColorHighlight}{true}

% If you would like to change the social information separator from a pipe (|) to something else
\renewcommand{\acvHeaderSocialSep}{\quad\textbar\quad}

\def\endfirstpage{\newpage}

%-------------------------------------------------------------------------------
%	PERSONAL INFORMATION
%	Comment any of the lines below if they are not required
%-------------------------------------------------------------------------------
% Available options: circle|rectangle,edge/noedge,left/right

\name{Juan David}{Leongómez Peña}

\position{Profesor Asociado}
\address{PhD University of Stirling · MSc University of Liverpool}

\mobile{(+57) 319 407 7102}
\email{\href{mailto:juanleongomez@gmail.com}{\nolinkurl{juanleongomez@gmail.com}}}
\homepage{jdleongomez.info}

% \gitlab{gitlab-id}
% \stackoverflow{SO-id}{SO-name}
% \skype{skype-id}
% \reddit{reddit-id}


\usepackage{booktabs}

\providecommand{\tightlist}{%
	\setlength{\itemsep}{0pt}\setlength{\parskip}{0pt}}

%------------------------------------------------------------------------------



% Pandoc CSL macros
\newlength{\cslhangindent}
\setlength{\cslhangindent}{1.5em}
\newlength{\csllabelwidth}
\setlength{\csllabelwidth}{2em}
\newenvironment{CSLReferences}[2] % #1 hanging-ident, #2 entry spacing
 {% don't indent paragraphs
  \setlength{\parindent}{0pt}
  % turn on hanging indent if param 1 is 1
  \ifodd #1 \everypar{\setlength{\hangindent}{\cslhangindent}}\ignorespaces\fi
  % set entry spacing
  \ifnum #2 > 0
  \setlength{\parskip}{#2\baselineskip}
  \fi
 }%
 {}
\usepackage{calc}
\newcommand{\CSLBlock}[1]{#1\hfill\break}
\newcommand{\CSLLeftMargin}[1]{\parbox[t]{\csllabelwidth}{\honortitlestyle{#1}}}
\newcommand{\CSLRightInline}[1]{\parbox[t]{\linewidth - \csllabelwidth}{\honordatestyle{#1}}}
\newcommand{\CSLIndent}[1]{\hspace{\cslhangindent}#1}

\begin{document}

% Print the header with above personal informations
% Give optional argument to change alignment(C: center, L: left, R: right)
\makecvheader

% Print the footer with 3 arguments(<left>, <center>, <right>)
% Leave any of these blank if they are not needed
% 2019-02-14 Chris Umphlett - add flexibility to the document name in footer, rather than have it be static Curriculum Vitae
\makecvfooter
  {19 de noviembre de 2023}
    {Juan David Leongómez Peña~~~·~~~Programa de curso}
  {\thepage}


%-------------------------------------------------------------------------------
%	CV/RESUME CONTENT
%	Each section is imported separately, open each file in turn to modify content
%------------------------------------------------------------------------------



\vspace{4mm}
\begin{tcolorbox}[enhanced,
        on line, 
        boxsep=4pt, left=0pt,right=0pt,top=0pt,bottom=0pt,
        colframe=white,colback=black]
  
\color{white}
\begin{LARGE}\begin{center}
Documento 3. \textbf{Programa de curso}
\end{center}\end{LARGE}
\end{tcolorbox}

\begin{LARGE}\begin{center}
\textbf{Proceso Básico: Lenguaje}\linebreak
Departamento de Psicología
\end{center}\end{LARGE}

\begin{cvskills}
  \cvskill
    {TIPO DE CURSO}
    {\textbf{Seminario}}
    
  \cvskill
    {PROFESOR}
    {Juan David Leongómez}
\end{cvskills}

\hypertarget{descripciuxf3n-general}{%
\section{Descripción General}\label{descripciuxf3n-general}}

\begin{footnotesize}
¿Por qué somos la única especie con un lenguaje verdadero? ¿Cómo y cuándo aparece el lenguaje? ¿Cómo adquieren el lenguaje los bebés? 

El lenguaje humano, una herramienta poderosa y fascinante, nos capacita para comunicarnos, expresar emociones, compartir conocimientos y moldear el mundo que nos rodea. A pesar de los asombrosos ejemplos de comunicación vocal en animales y los intrigantes casos de mensajes acústicos complejos, e incluso la existencia de proto-sintaxis, proto-semántica y evolución cultural, ninguna otra especie alcanza la complejidad del lenguaje humano. 
Este seminario tiene como propósito explorar diversos aspectos del estudio del lenguaje, desde una perspectiva transdisciplinar, centrándose en tres aspectos fundamentales: (1) comunicación vocal y antecedentes del lenguaje, (2) hipótesis sobre el origen del lenguaje, y (3) adquisición y procesamiento del lenguaje.

A través de este programa, además de adquirir una visión amplia sobre el fenómeno del lenguaje, se fortalecerá la formación científica e investigativa, ya que el curso permitirá a las y los estudiantes adquirir herramientas de lectura crítica de artículos científicos —incluyendo la evaluación de la calidad de su evidencia—, y mejorar la capacidad de sintetizar ideas, aprovechando que el estudio científico del lenguaje es un campo complejo en el que rara vez hay respuestas concretas. 

Dada la amplitud de la literatura en este campo, este curso ofrecerá una perspectiva interdisciplinaria y general, abordando artículos que fusionan métodos y evidencia desde los enfoque de la psicología, las ciencias cognitivas y las neurociencias, la biología evolutiva, la etología, la lingüística, la antropología y la psiquiatría, entre otras disciplinas. Además, puesto que la literatura comprende desde artículos empíricos hasta revisiones y propuestas teóricas, este curso integrará conocimientos adquiridos en otros cursos, incluyendo análisis estadísticos y habilidades analíticas.
\end{footnotesize}

\hypertarget{resultados-de-aprendizaje}{%
\section{Resultados de Aprendizaje}\label{resultados-de-aprendizaje}}

\begin{footnotesize}

Al finalizar el curso, las y los estudiantes estarán en capacidad de:

\par
\begingroup
\leftskip3em
\rightskip\leftskip

1. Identificar y explicar las características únicas del lenguaje humano en comparación con la comunicación vocal en otras especies animales (Demuestra \textbf{Comprensión de la singularidad del lenguaje humano})

2. Analizar y evaluar críticamente diversas hipótesis sobre el origen del lenguaje, entendiendo que no hay respuestas únicas ni \textit{verdaderas} (Demuestra \textbf{Entendimiento de las teorías del origen del lenguaje})

3. Comprender que hay tanto evidencia de recursos neuronales compartidos, como de especificidad de dominio, entre la música y el lenguaje, a partir de la evidencia que proveen algunos desórdenes del desarrollo y lesiones cerebrales específicas (Demuestra \textbf{Conocimiento sobre la relación entre lenguaje y música})

4. Describir los procesos cognitivos implicados en la adquisición y procesamiento del lenguaje, incluyendo aspectos relacionados con la percepción, la gramática y la comprensión (Demuestra \textbf{Comprensión del procesamiento del lenguaje})

5. Demostrar habilidades para analizar, sintetizar y evaluar críticamente la literatura científica relacionada con el lenguaje, expresando y defendiendo sus puntos de vista a través de ensayos y discusiones basados en evidencia (Demuestra \textbf{Desarrollo de habilidades críticas})

6. Reconocer que el abordaje de temas complejos como el lenguaje requiere de miradas desde diferentes disciplinas (Demuestra \textbf{Entendimiento de la necesidad de miradas transdisciplinares para abordar temas complejos})

\par
\endgroup

\end{footnotesize}

\hypertarget{metodologuxeda}{%
\section{Metodología}\label{metodologuxeda}}

\begin{footnotesize}
Dado que el estudio del lenguaje es un campo transdisciplinar y complejo, en el que rara vez hay respuestas concretas, el formato de seminario es ideal, pues permite no sólo abordar la complejidad de los temas desde el debate, sino también integrar conocimientos de otros cursos y acercarse a las miradas desde otras disciplinas. 

Por esto, este curso se basará en una versión simplificada del método de \textit{seminario alemán}, adaptada para el nivel de formación. Bajo la dirección del profesor, se abordarán temáticas específicas en cada sesión, a partir de la lectura y discusión de artículos relevantes. 
\end{footnotesize}

\hypertarget{organizaciuxf3n-del-curso-por-muxf3dulos}{%
\subsection{Organización del curso por
módulos}\label{organizaciuxf3n-del-curso-por-muxf3dulos}}

\begin{footnotesize}
El curso se inicia con una cátedra a cargo del profesor, que tiene como objetivo no solo presentar el curso y su metodología, sino también proporcionar una visión general de los aspectos básicos de la evolución y el significado de las señales vocales, necesarios para comprender algunos puntos clave de la literatura. 

Posteriormente, las siguientes 15 sesiones se dividen en tres módulos, centrados en los aspectos fundamentales propuestos: (1) comunicación vocal y antecedentes del lenguaje, (2) hipótesis sobre el origen del lenguaje, y (3) adquisición y procesamiento del lenguaje.

Cada módulo consta de cinco sesiones: las primeras cuatro abordan temas específicos de cada aspecto mediante la exposición y discusión de artículos relevantes. La última sesión de cada módulo se dedica a una discusión general basada en las conclusiones y preguntas que los estudiantes hayan extraído durante la preparación y escritura de ensayos que prepararán (incluyendo los artículos adicionales que hayan leído).
\end{footnotesize}

\hypertarget{organizaciuxf3n-de-cada-sesiuxf3n}{%
\subsection{Organización de cada
sesión}\label{organizaciuxf3n-de-cada-sesiuxf3n}}

\begin{footnotesize}
Aunque todos los estudiantes deben realizar una lectura general de algunos artículos, para cada sesión se asignará a algunos estudiantes la lectura en profundidad de uno de los artículos. Estos estudiantes recibirán el apoyo del profesor para preparar (1) una exposición del artículo y (2) una actividad ilustrativa y didáctica sobre sus contenidos (por ejemplo, un \textit{pub quiz} o un concurso de interpretación de resultados).

Al concluir las exposiciones, se llevará a cabo una discusión sobre los resultados e implicaciones del tema de la sesión, guiada por el profesor. Se buscará que los estudiantes exploren las limitaciones, fortalezas y explicaciones alternativas, profundizando tanto en las bases teóricas como en las fortalezas y limitaciones analíticas y metodológicas de cada artículo, con el objetivo de fomentar la lectura crítica.
\end{footnotesize}

\hypertarget{evaluaciuxf3n}{%
\section{Evaluación}\label{evaluaciuxf3n}}

\begin{footnotesize}
La evaluación en este seminario se centra en la presentación y discusión de artículos clave de la literatura. Se valorará la capacidad de los estudiantes para sintetizar los artículos, así como la creación de actividades ilustrativas y didácticas relacionadas con su contenido, y su participación en las discusiones subsiguientes.

Además, a lo largo del semestre, se requerirá que los estudiantes entreguen tres ensayos (uno por cada módulo) que relacionen elementos de varios artículos y disciplinas sobre los temas tratados. Estos ensayos permitirán que cada estudiante presente sus propias conexiones y conclusiones, y que busque literatura adicional para respaldar sus ideas con evidencia.
\end{footnotesize}

\hypertarget{distribuciuxf3n-y-pesos-de-cada-actividad}}{}{\begin{cvitems}
\item Exposición y actividad ilustrativa planeada: \textbf{10\%}
\item Intervenciones en discusiones: \textbf{5\%}
\item Ensayo: \textbf{15\%}
\end{cvitems}}
    \cventry{\textit{Hipótesis sobre el origen del lenguaje}}{Segundo corte}{\textbf{Peso 30\%}}{}{\begin{cvitems}
\item Exposición y actividad ilustrativa planeada: \textbf{10\%}
\item Intervenciones en discusiones: \textbf{5\%}
\item Ensayo: \textbf{15\%}
\end{cvitems}}
    \cventry{\textit{Adquisición y procesamiento del lenguaje}}{Tercer corte}{\textbf{Peso 40\%}}{}{\begin{cvitems}
\item Exposición y actividad ilustrativa planeada: \textbf{10\%}
\item Intervenciones en discusiones: \textbf{10\%}
\item Ensayo: \textbf{20\%}
\end{cvitems}}
\end{cventries}

\hypertarget{estructura-del-curso}{%
\section{Estructura del Curso}\label{estructura-del-curso}}

\begin{footnotesize}
A continuación se presenta la estructura del curso, incluyendo el tema de cada sesión. 

El número de artículos propuestos como base para cada sesión varía en función de su complejidad y/o extensión. 

De acuerdo al número de estudiantes en el curso, las exposiciones y actividades ilustrativas sobre los artículos serán preparadas en grupos de un tamaño acorde, y de ser necesario se excluirán (o aumentarán) artículos a algunas sesiones.  
\end{footnotesize}

\begingroup\fontsize{7.5}{9.5}\selectfont

\begin{longtable}{|>{\raggedleft\arraybackslash}p{4em}|>{\raggedright\arraybackslash}p{30em}|>{\raggedright\arraybackslash}p{12em}|>{\raggedright\arraybackslash}p{12em}|}
\hline
\multicolumn{1}{c}{\begingroup\fontsize{11}{13}\selectfont \em{\textbf{Semana}}\endgroup} & \multicolumn{1}{c}{\begingroup\fontsize{11}{13}\selectfont \em{\textbf{Tema}}\endgroup} & \multicolumn{1}{c}{\begingroup\fontsize{11}{13}\selectfont \em{\textbf{Lecturas}}\endgroup} & \multicolumn{1}{c}{\begingroup\fontsize{11}{13}\selectfont \em{\textbf{Trabajo autónomo}}\endgroup}\\
\hline
\endfirsthead
\multicolumn{4}{@{}l}{\textit{(continued)}}\\
\hline
\multicolumn{1}{c}{\begingroup\fontsize{11}{13}\selectfont \em{\textbf{Semana}}\endgroup} & \multicolumn{1}{c}{\begingroup\fontsize{11}{13}\selectfont \em{\textbf{Tema}}\endgroup} & \multicolumn{1}{c}{\begingroup\fontsize{11}{13}\selectfont \em{\textbf{Lecturas}}\endgroup} & \multicolumn{1}{c}{\begingroup\fontsize{11}{13}\selectfont \em{\textbf{Trabajo autónomo}}\endgroup}\\
\hline
\endhead
1 & Introducción y presentación del curso\linebreak \textbf{Cátedra sobre evolución y significado de las señales vocales} & — & —\\
\hline
\multicolumn{4}{l}{\cellcolor{black}{\textcolor{white}{\textbf{MÓDULO 1: \textit{Comunicación vocal y antecedentes del lenguaje}}}}}\\
\hline
\hspace{1em}2 & Percepción a partir de la voz & \cite{beeMaleGreenFrogs2000}\linebreak \cite{charltonContextrelatedAcousticVariation2011}\linebreak \cite{RefWorks:723}\linebreak \cite{Collins2000}\linebreak \cite{collinsVocalVisualAttractiveness2003}\linebreak \cite{putsDominanceEvolutionSexual2006}\linebreak \cite{putsSexualSelectionMale2016} & Lectura de artículos\linebreak (Preparación exposición y actividad)*\\
\hline
\hspace{1em}3 & Precursores del lenguaje: proto-semántica & \cite{evansChickenFoodCalls1999}\linebreak \cite{greeneRedSquirrelsTamiasciurus1998}\linebreak \cite{seyfarthMonkeyResponsesThree1980} & Lectura de artículos\linebreak (Preparación exposición y actividad)*\\
\hline
\hspace{1em}4 & Precursores del lenguaje: proto-sintaxis & \cite{marlerSpeciesuniversalMicrostructureLearned1984}\linebreak \cite{podosPermissivenessLearningDevelopment1999}\linebreak \cite{zuberbuhlerSyntaxCompositionalityAnimal2019} & Lectura de artículos\linebreak (Preparación exposición y actividad)*\\
\hline
\hspace{1em}5 & Evolución cultural de la comunicación vocal & \cite{eriksenCulturalChangeSongs2005}\linebreak \cite{lutherUrbanNoiseCultural2010}\linebreak \cite{noadCulturalRevolutionWhale2000} & Lectura de artículos\linebreak (Preparación exposición y actividad)*\\
\hline
6 & \textbf{ENTREGA ENSAYO:} ¿Qué diferencias y similitudes existen entre el lenguaje humano y la comunicación vocal en otras especies animales?\linebreak
\hspace{1em}\cellcolor{lightgray2}{\textbf{Discusión general sobre comunicación vocal y antecedentes animales del lenguaje a partir de los ensayos}} & \cellcolor{lightgray2}{—} & \cellcolor{lightgray2}{Preparación ensayo}\\
\hline
\multicolumn{4}{l}{\cellcolor{black}{\textcolor{white}{\textbf{MÓDULO 2: \textit{Hipótesis sobre el origen del lenguaje}}}}}\\
\hline
\hspace{1em}7 & Expresiones faciales y gestualidad & \cite{mccombCoevolutionVocalCommunication2005}\linebreak \cite{pollickApeGesturesLanguage2007} & Lectura de artículos\linebreak (Preparación exposición y actividad)*\\
\hline
\hspace{1em}8 & Hipótesis de un solo paso (Chomsky) y La hipótesis de Rómulo y Remo & \cite{chomskyThreeFactorsLanguage2005}\linebreak \cite{vyshedskiyLanguageEvolutionRevolution2019} & Lectura de artículos\linebreak (Preparación exposición y actividad)*\\
\hline
\hspace{1em}9 & Acicalamiento social (Dunbar) y \textit{From-where-to-what theory} & \cite{dunbarOriginSubsequentEvolution2003}\linebreak \cite{dunbarCoevolutionNeocorticalSize1993a}\linebreak \cite{polivaWhereWhatNeuroanatomically2017} & Lectura de artículos\linebreak (Preparación exposición y actividad)*\\
\hline
\hspace{1em}10 & Comunicación con bebés prelingüísticos y musicalidad & \cite{falkPrelinguisticEvolutionEarly2005}\linebreak \cite{leongomezMusicalityHumanVocal2022} & Lectura de artículos\linebreak (Preparación exposición y actividad)*\\
\hline
11 & \textbf{ENTREGA ENSAYO:} ¿Cuál crees que es la hipótesis que mejor exoplica el origen del lenguaje?\linebreak
\hspace{1em}\cellcolor{lightgray2}{\textbf{Discusión general sobre hipótesis sobre el origen del lenguaje a partir de los ensayos}} & \cellcolor{lightgray2}{—} & \cellcolor{lightgray2}{Preparación ensayo}\\
\hline
\multicolumn{4}{l}{\cellcolor{black}{\textcolor{white}{\textbf{MÓDULO 3: \textit{Adquisición y procesamiento del lenguaje}}}}}\\
\hline
\hspace{1em}12 & Maternés: comunicación emocional & \cite{smithInfantDirectedSpeechModulated2008}\linebreak \cite{RefWorks:464}\linebreak \cite{hiltonAcousticRegularitiesInfantdirected2022} & Lectura de artículos\linebreak (Preparación exposición y actividad)*\\
\hline
\hspace{1em}13 & Adquicisión del lenguaje & \cite{thiessenInfantDirectedSpeechFacilitates2005}\linebreak \cite{golinkoffBabyTalkMe2015}\linebreak \cite{cristiaInputLanguagePhonetics2013}\linebreak \cite{kuhlBrainMechanismsEarly2010}\linebreak \cite{eigstiLanguageAcquisitionAutism2011} & Lectura de artículos\linebreak (Preparación exposición y actividad)*\\
\hline
\hspace{1em}14 & Relación entre música y lenguaje: evidencia de recursos neuronales compartidos y especificidad de dominio (daños cerebrales y desórdenes del desarrollo) & \cite{sammlerOverlapMusicalLinguistic2009}\linebreak \cite{koelschAdultsChildrenProcessing2005}\linebreak \cite{coumelSecondLanguageAccent2019}\linebreak \cite{zuberbuhlerSyntaxCompositionalityAnimal2019}\linebreak \cite{jentschkeChildrenSpecificLanguage2008}\linebreak \cite{pearceSelectedObservationsAmusia2005}\linebreak \cite{signoretAphasiaAmusiaBlind1987} & Lectura de artículos\linebreak (Preparación exposición y actividad)*\\
\hline
\hspace{1em}15 & Procesamiento lingüístico & \cite{campbellSignLanguageBrain2008}\linebreak \cite{liNeuroplasticityFunctionSecond2014}\linebreak \cite{friedericiBrainBasisLanguage2011} & Lectura de artículos\linebreak (Preparación exposición y actividad)*\\
\hline
16 & \textbf{ENTREGA ENSAYO:} Tema libre \linebreak
\hspace{1em}\cellcolor{lightgray2}{\textbf{Discusión general sobre adquisición y procesamiento del lenguaje a partir de los ensayos}} & \cellcolor{lightgray2}{—} & \cellcolor{lightgray2}{Lectura de artículos\linebreak (Preparación exposición y actividad)*}\\
\hline
\multicolumn{4}{l}{\rule{0pt}{1em}*Para estudiantes a cargo de exponer un artículo en esa sesión.}\\
\end{longtable}
\endgroup{}

\hypertarget{referencias}{%
\section{Referencias}\label{referencias}}

\begin{multicols}{2}
\AtNextBibliography{\footnotesize}
\printbibliography[heading=none]
\end{multicols}



\end{document}
